% !TEX root = main.tex





\section{IMU pre-integration in S3 and SO(3)}
\label{sec:imu}

%%%%%%%%%%%%%%%%%%%%%%%%%%%%%%%%%%%%%%%%%%%%%%%%%%%%%%%%%%%%
\subsection{Overview}

Due to the different rates of IMU data and keyframe (KF) creation, hundreds of IMU measurements need to be integrated to generate a motion factor linking two consecutive KFs. 
The integration of the motion equations in an absolute reference frame strongly depends on the initial conditions of orientation, velocity and IMU bias.
Therefore, the changes in the estimates of these magnitudes (inherent to the iterative nature of the optimization) affect the whole motion integral. 
Delta pre-integration theory was developed \cite{LUPTON-09,forster2015imu} to avoid the need of re-integrating all IMU data at each iteration. 
On one hand, this theory defines new motion magnitudes called \emph{deltas}, which are independent of the initial conditions of orientation and velocity, and thus depend \emph{only} on the IMU data and bias. 
On the other hand, the effect of the changes in the bias estimates is linearized so that the deltas can be corrected a posteriori using pre-computed Jacobians. 

In this section, we revise the IMU-preintegration theory, providing three contributions: 
1) a segmentation of the computation pipeline (from measurements to body magnitudes, to the current delta, and to the integrated delta); 
2) a physical interpretation of the delta magnitudes; 
and 
3) a simpler yet rigorous algebraic approach, valid for both the S3 (quaternion) and SO3 (rotation matrix) manifolds, which takes profit of the pipeline segmentation and the chain rule to compute the otherwise combersome \cite{forster2015imu} Jacobians. 
Important complements are provided in \appRef{sec:derivatives_SO3} for the sake of completeness.

%%%%%%%%%%%%%%%%%%%%%%%%%%%%%%%%%%%%%%%%%%%%%%%%%%%%%%%%%%%%
\subsection{Quaternion rotations}

We define the quaternion-by-vector product $\od$ so that
%
\begin{align}
\bfq\od\bfv \te \bfq\ot\bfv\ot\bfq^*
~,
\end{align}
%
where the symbol $\ot$ indicates the quaternion product.
That is, quaternion-by-vector products using the symbol $\od$ perform 3D rotations. 
Notice that if $\bfR$ is the rotation matrix equivalent to the quaternion $\bfq$, then 
%
$\bfq\od\bfv = \bfR\,\bfv$. 
%
This straightforward equivalence
allows us to define all the forthcoming IMU-preintegration algebra in a way that allows a direct passage between the $S3$ (quaternion) and $SO(3)$ (rotation matrix) implementations.
%Since the quaternion implementation is one of the contributions, in the following we will refer to the groups $S3$ and $SO(3)$ with the unique name $S3$. 




%%%%%%%%%%%%%%%%%%%%%%%%%%%%%%%%%%%%%%%%%%%%%%%%%%%%%%%%%%%%

\subsection{State integration in the absolute reference frame}

We define the world-referenced states of position, velocity, and orientation quaternion, $\bfx\!=\!(\bfp,\bfv,\bfq)$. 
Their time evolution is governed by the kinematic equation,
%
\begin{align}\label{equ:cont_basic}
\begin{split}
\dot\bfp &= \bfv \\
\dot\bfv &= \bfg + \bfq\od\bfa \\
\dot\bfq &= \frac12\bfq\ot\bw 
\end{split}
\end{align}
%
where we identify $\bfb=(\bfa,\bw)$ as the \emph{body magnitudes}, that is, the magnitudes of acceleration and angular rate observed at the IMU reference frame. These are obtained at discrete times $t_j$ from biased and noisy IMU measurements, \ie,
%
\begin{align}\label{equ:body}
\begin{split}
\bfa_j &= \bfa_{m,j} - \bfa_{b,j} - \bfa_n \\
\bw_j  &= \bw_{m,j}  - \bw_{b,j}  - \bw_n 
~,
\end{split}
\end{align}
%
with $\bullet_m$ the measurements, $\bullet_b$ the biases, and $\bullet_n$ the noises.
Assuming constant body magnitudes within the IMU sampling period $\dt\te t_k-t_j$, we have the discrete-time relation
%
\begin{align}\label{equ:integration_state}
\begin{split}
\bfp_{k} &= \bfp_j + \bfv_j\dt  + \frac12\bfg\dt^2 + \frac12\bfq_j\od\bfa_j\dt^2 \\
\bfv_{k} &= \bfv_j + \bfg\dt + \bfq_j\odot\bfa_j\dt \\
\bfq_{k} &= \bfq_j\ot\Exp(\bw_j\dt/2) 
\end{split}
\end{align}


%%%%%%%%%%%%%%%%%%%%%%%%%%%%%%%%%%%%%%%%%%%%%%%%%%%%%%%%%%%%
\subsection{Delta definitions}

Consider a non-rotating reference frame that is free-falling at the acceleration of gravity $\bfg$, and name it $\cG_t$. 
An ideal (unbiased and noiseless) IMU glued to this frame would measure null accelerations and rotations. 
Any non-null measurements would imply motion of the IMU \wrt  $\cG_t$.

At $t=t_i$, we initialize $\cG_i$ at $\bfx_i=(\bfq_i,\bfv_i,\bfq_i)$.
At a later time $t=t_j$,  $\cG_j$ has fallen according to $\bfg$, and the state of our moving body is at $\bfx_j=(\bfq_j,\bfv_j,\bfq_j)$.
The motion delta, denoted $\D_{ij}%=\D(\bfx_i,\bfx_j)
$, is defined as the state (position, velocity, orientation) of our body \wrt  $\cG_j$, that is,
%
\begin{align}\label{equ:delta_definition}
\begin{split}
\Dp_{ij} &= \bfq_i^*\od\Big(\bfp_j - \bfp_i - \bfv_i\Dt_{ij} - \frac12\bfg\Dt_{ij}^2\Big) \\
\Dv_{ij} &= \bfq_i^*\od(\bfv_j - \bfv_i - \bfg\Dt_{ij}) \\
\Dq_{ij} &= \bfq_i^*\ot\bfq_j 
\end{split}
\end{align}
%
where $\Dt_{ij} \te t_j - t_i$. 
Notice that this definition is the same provided in \cite{LUPTON-09,FORSTER-16-techrep}, and that we have given it a clear physical meaning.
Interestingly, the deltas form a group under the composition $\D_{ik}\te\D_{ij}\oplus\D_{jk}$, defined by,
%
\begin{align} \label{equ:composition}
\begin{split}
\Dp_{ik} 
&= \Dp_{ij} + \Dv_{ij}\Dt_{jk} + \Dq_{ij}\od\Dp_{jk} \\
\Dv_{ik} 
&= \Dv_{ij} + \Dq_{ij}\od\Dv_{jk} \\
\Dq_{ik} 
&= \Dq_{ij}\ot\Dq_{jk} 
\end{split}
\end{align}
%
with identity $\D_0=[(0,0,0),(0,0,0),(1,0,0,0)]$, and inverse $\D_{ji}\te\D_{ij}\inv$ shuch that $\D\inv\op\D=\D\oplus\D\inv=\D_0$ (the inverse expression is not given for space reasons).
At any time $j$ we can recover the state estimate $\bfx_j$ given the state estimate $\bfx_i$ and the delta $\D_{ij}$,
%
\begin{align} \label{equ:reconstruction}
\begin{split}
\bfp_j &= \bfp_i + \bfv_i\Dt_{ij} + \frac12\bfg\Dt_{ij}^2 + \bfq_i\od\Dp_{ij} \\
\bfv_j &= \bfv_i + \bfg\Dt_{ij} + \bfq_i\od\Dv_{ij} \\
\bfq_j &= \bfq_i\ot\Dq_{ij}   
\end{split}
\end{align}

%%%%%%%%%%%%%%%%%%%%%%%%%%%%%%%%%%%%%%%%%%%%%%%%%%%%%%%%%%%%
\subsection{Incremental delta pre-integration}

Substituting the integration equation \eqRef{equ:integration_state} in the delta definitions \eqRef{equ:delta_definition}, we obtain the incremental delta pre-integration,
%
\begin{align}\label{equ:g_second_order_pre-integration}
\begin{split}
\Dp_{ik} 
&= \Dp_{ij} + \Dv_{ij}\dt + \frac12\Dq_{ij}\od\bfa_j\dt^2 \\
\Dv_{ik} 
&= \Dv_{ij} + \Dq_{ij}\od\bfa_j\dt \\
\Dq_{ik} 
&= \Dq_{ij}\ot\Exp(\bw_j\dt) 
\end{split}
\end{align}
%
which starts at $\D_{ii}=\D_0$. Interestingly, \eqRef{equ:g_second_order_pre-integration} is analogous to the motion of a body \emph{in an inertial frame} under constant acceleration and rotation rate.
Notice that by letting the reference frame fall with gravity, we get rid of the dependence on gravity of the integration equations, and the only magnitudes entering the integral are the body magnitudes.
Notice also that we can define a proper delta $\delta_{jk}$ from the current body magnitudes $\bfb_j=(\bfa_j,\bw_j) \te \bfb_{m,j} - \bfb_{b,i} - \bfb_{n,j}$ at time $t_j$,
%
\begin{align}\label{equ:delta_creation}
\begin{split}
\dpp_{jk} &= \frac12\bfa_j\dt^2 \\
\dv_{jk} &= \bfa_j\dt \\
\dq_{jk} &= \Exp(\bw_j\dt)
\end{split}
\end{align}
%
and write the integration \eqRef{equ:g_second_order_pre-integration} as the composition 
%
\begin{align}\label{equ:composition_compact}
\D_{ik}=\D_{ij}\oplus\delta_{jk}
\end{align}
%
described in \eqRef{equ:composition}. In the following, we will identify $\D$ with the pre-integrated delta, and $\delta$ with the current delta.


%%%%%%%%%%%%%%%%%%%%%%%%%%%%%%%%%%%%%%%%%%%%%%%%%%%%%%%%%%%%
\subsection{Jacobians}

\newcommand{\jac}[2]{{\bfJ^{#1}_{#2}}}

Notation: 
We note all Jacobians with $\jac{y}{x}\te\dparil{y}{x}$. 
Please refer to \appRef{sec:derivatives_SO3} for details on the development of all non-trivial Jacobian blocks in this section.



\subsubsection{Jacobians of the body magnitudes}

We have from \eqRef{equ:body},
%
\begin{align}\label{equ:jac_body}
\jac{\bfb}{\bfb_m}&=\bfI_6 & \jac{\bfb}{\bfb_b}&=-\bfI_6 & \jac{\bfb}{\bfb_n}&=-\bfI_6
~.
\end{align}

\subsubsection{Jacobians of the current delta}
\label{sec:jac_data}

We have from \eqRef{equ:delta_creation},
%
\begin{align}\label{equ:jac_current_delta}
\jac{\delta jk}{\bfb j} =
\begin{bmatrix}
\frac12\bfI\dt^2 	& \bf0 \\
\bfI\dt 			& \bf0 \\
\bf0 	          & \bfJ_r(\bw_j\dt)\dt
\end{bmatrix} 
\in \bbR^{9\tcross6}
\end{align}
%
where we develop the lower-right block as in \appRef{sec:jac_R3toSO3}.





\subsubsection{Jacobians of the delta composition}

We differentiate the delta composition \eqRef{equ:composition_compact} described in \eqRef{equ:composition},
%
\begin{subequations}\label{equ:jac_composition}
\begin{align}
\jac{\D ik}{\D ij} &= \begin{bmatrix}
\bfI  & \bfI\dt & - \DR_{ij}  \hatx{\dpp_{jk}}  \\
\bf0  & \bfI    & - \DR_{ij}  \hatx{\dv_{jk}} \\
\bf0  & \bf0    &   \dR_{jk}\tr 
\end{bmatrix} 
\in \bbR^{9\tcross9}
\\
\jac{\D ik}{\delta jk} &= \begin{bmatrix}
\DR_{ij} & \bf0     & \bf0 \\
\bf0     & \DR_{ij} & \bf0 \\
\bf0     & \bf0     & \bfI  
\end{bmatrix}
~~~~~~~~~\in \bbR^{9\tcross9}
\end{align}
\end{subequations}
%
where $\DR_{ij}$ and $\dR_{jk}$ are the rotation matrix deltas corresponding to the respective quaternion deltas $\Dq_{ij}$ and $\dq_{jk}$. 
We develop all the non-trivial blocks as in \appRef{sec:jac_SO3xR3toR3}.


%%%%%%%%%%%%%%%%%%%%%%%%%%%%%%%%%%%%%%%%%%%%%%%%%%%%%%%%%%%%
\subsection{Incremental delta covariance integration}

Let $\bfQ_\D$ be the covariance of the pre-integrated delta, and $\bfQ_n$ that of the measurement noise. For convenience, we first compute the covariance of the current delta,
%
\begin{align}
\bfQ_\delta =  \jac{\delta}{\bfb_n}\bfQ_n\,\jac{\delta}{\bfb_n}\tr
~,
\end{align}
%
where $\jac{\delta}{\bfb_n} = \jac{\delta}{\bfb} \cdot \jac{\bfb}{\bfb_n}
$ is the noise Jacobian, obtained with \eqsRef{equ:jac_body}{equ:jac_current_delta} and the chain rule.
The delta covariance is then integrated with
%
\begin{align}
\bfQ_{\D ik} = \jac{\D ik}{\D ij}\bfQ_{\D ij}\,\jac{\D ik}{\D ij}\tr + \jac{\D ik}{\delta jk}\bfQ_\delta\,\jac{\D ik}{\delta jk}\tr
~,
\end{align}
%
using Jacobians \eqRef{equ:jac_composition}, and starting at $\bfQ_{\D ii}={\bf0}_{9\tcross9}$.



%%%%%%%%%%%%%%%%%%%%%%%%%%%%%%%%%%%%%%%%%%%%%%%%%%%%%%%%%%%%
\subsection{Delta correction with new bias}

Let $\ol\D$ and $\ol{\bfb_b}$ be respectively the pre-integrated delta and the bias values used during pre-integration. Since the bias estimates change at each iteration of the optimizer, we need to update the delta according to the new bias values $\bfb_b$. We do so with the linearized update,
%
\begin{align}\label{equ:delta_correction}
\D = \ol\D + \jac{\D}{\bfb_b}(\bfb_b - \ol{\bfb_b})
~,
\end{align}
%
where $\jac{\D}{\bfb_b}$ is the pre-integrated bias Jacobian, computed incrementally using also the chain rule,
%
\begin{align*}
\jac{\D_{ik}}{\bfb_b} 
&= 
\jac{\D_{ik}}{\D_{ij}} \cdot \jac{\D_{ij}}{\bfb} \cdot \jac{\bfb}{\bfb_b} 
+ 
\jac{\D_{ik}}{\delta_{jk}} \cdot \jac{\delta_{jk}}{\bfb} \cdot \jac{\bfb}{\bfb_b} 
%\\
%&= \jac{\D_{ik}}{\D_{ij}} \cdot \jac{\D_{ij}}{\bfb_b} - \jac{\D_{ik}}{\delta} \cdot \jac{\delta}{\bfb}
~,
\end{align*}
%
which we might note for extra clarity,
%
\begin{align}
\jac{\D}{\bfb_b}|_{ik} = \jac{\D ik}{\D ij}\jac{\D}{\bfb_b}|_{ij} - \jac{\D ik}{\delta jk}\jac{\delta jk}{\bfb_b}
~.
\end{align}
%
The Jacobian integral starts at $\jac{\D}{\bfb_b}|_{ii} = {\bf0}_{9\tcross 9}$.


%%%%%%%%%%%%%%%%%%%%%%%%%%%%%%%%%%%%%%%%%%%%%%%%%%%%%%%%%%%%
\subsection{Residual}

The computation of the residuals for the IMU delta factors (see \figRef{fig:factor_graph}, green) requires: the state estimates $\bfx_i$ and $\bfx_j$; the current bias estimates $\bfb_{b,i}$; the pre-integrated delta $\ol{\D_{ij}}$; the bias used during pre-integration $\ol{\bfb_{b,i}}$; and the pre-integrated bias Jacobian $\jac{\D}{\bfb_b}$. The process is best understood if split into smaller steps: we first compute a corrected delta $\D_{ij}$ using \eqRef{equ:delta_correction}; then compute a predicted delta $\widehat\D_{ij}$ using \eqRef{equ:delta_definition}; and finally compute the residual with
%
\begin{align}\label{equ:imu_residual}
\bfr(\bfx_i,\bfx_j,\bfa_{b,i},\bw_{b,i}) 
= \begin{bmatrix}
\Dp_{ij}-\widehat\Dp_{ij} \\
\Dv_{ij}-\widehat\Dv_{ij} \\
\Log(\widehat\Dq_{ij}^*\ot\Dq_{ij})
\end{bmatrix} 
\in \bbR^9
~.
\end{align}
%
Its information matrix is given by $\bfOmega=\bfQ_{\D ik}\inv$.






