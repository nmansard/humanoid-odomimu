% !TEX root = main.tex


% More macros
\newcommand{\bw}{{\bfomega}}
\newcommand{\bth}{{\bftheta}}
\newcommand{\bphi}{{\bfphi}}
\newcommand{\nth}{\norm{\bth}}
\newcommand{\ab}{{\bfa_b}}
\newcommand{\wb}{{\bw_b}}
\newcommand{\D}{\Delta}
\newcommand{\Dzero}{{\D^0}}
\newcommand{\Dp}{{\D\bfp}}
\newcommand{\Dv}{{\D\bfv}}
\newcommand{\Dth}{{\D\bth}}
\newcommand{\Dq}{{\D\bfq}}
\newcommand{\DR}{{\D\bfR}}
\newcommand{\DP}{{\D\bfP}}
\newcommand{\DV}{{\D\bfV}}
\newcommand{\DTH}{{\D\bfTheta}}
\newcommand{\Dw}{{\D\bw}}
\newcommand{\DW}{{\D\bfOmega}}
\newcommand{\dpp}{{\delta\bfp}}
\newcommand{\dv}{{\delta\bfv}}
\newcommand{\dth}{{\delta\bth}}
\newcommand{\dq}{{\delta\bfq}}
\newcommand{\dR}{{\delta\bfR}}
\newcommand{\dP}{{\delta\bfP}}
\newcommand{\dV}{{\delta\bfV}}
\newcommand{\dTH}{{\delta\bfTheta}}
\newcommand{\dw}{{\delta\bw}}

\newcommand{\te}{\triangleq}
\newcommand{\od}{\odot}

% Helper /keys/ 
\newcommand{\tcom}[1]{{\footnotesize/\texttt{#1}/} }
\newcommand{\com}[1]{{\footnotesize/\texttt{#1}/~} }
\newcommand{\cdef}{\com{def}}
\newcommand{\cchain}{\com{chain}}
\newcommand{\ccross}{\com{cross}}
\newcommand{\cJr}{\com{Jr}}
\newcommand{\csmall}{\com{small}}
\newcommand{\cswap}{\com{swap}}
\newcommand{\ctrans}{\com{trans}}
\newcommand{\clog}{\com{Log}}
\newcommand{\clim}{\com{lim}}
\newcommand{\ccancel}{\com{cancel}}
\newcommand{\cexpand}{\com{expand}}
\newcommand{\csubst}{\com{subst}}

\section{IMU pre-integration in S3 and SO(3)}
\subsection{Quaternion rotations}

We define the quaternion-by-vector product $\od$ so that
%
\begin{align}
\bfq\od\bfv \te \bfq\ot\bfv\ot\bfq^*
~,
\end{align}
%
where the symbol $\ot$ indicates the quaternion product.
That is, quaternion-by-vector products using the symbol $\od$ perform 3D rotations. 
%This product satisfies,
%%
%\begin{align*}
%\bfq_2\od(\bfq_1\od\bfv) &= (\bfq_2\ot\bfq_1)\od\bfv
%\end{align*}
%%
%that is, the group of quaternions $(\bfq\in\bbH,\ot)$ is a transformation group acting on 3D vectors through the rotation action $\od$.
Notice that if $\bfR$ is the rotation matrix equivalent to the quaternion $\bfq$, then $\bfq\od\bfv = \bfR\,\bfv$ and $\bfq^*\od\bfv = \bfR\tr\,\bfv$. 
In practice, the operator $\od$ may be implemented as ~$\bfq\od\bfv = \bfq\ot\bfv\ot\bfq^*$ ~or~ $\bfq\od\bfv = \bfR\,\bfv$, whichever is faster.

\subsection{Exp and Log maps}

We use vectorized versions of the exponential and logarithmic maps in $S3$ (the group of unit quaternions) and $SO(3)$ (the group of rotation 3-matrices), and mark them with capitalized names $\Exp()$ and $\Log()$. They operate directly on the vector space $\bbR^3$, and use either quaternions as the representation of $S3$,
%
\begin{subequations}
\begin{align}
\bfq
%\{\bth\} 
&= \Exp(\bth) \te \begin{bmatrix}
\cos(\theta/2) \\ \bfu\sin(\theta/2)
\end{bmatrix}\\ 
\theta\bfu &= \Log(\bfq) \te 2\,\qv\frac{\arctan({\norm{\qv},q_w})}{\norm{\qv}}
~,
\end{align}
\end{subequations}
%
or rotation matrices as the representation of $SO(3)$, 
%
\begin{subequations}
\begin{align}
\bfR
%\{\bth\} 
&= \Exp(\bth) \te \bfI + \sin\theta\hatx{\bfu} + (1-\cos\theta)\hatx{\bfu}^2~ \label{equ:rodrigues} \\ 
\theta\bfu &= \Log(\bfR) \te \frac{\theta(\bfR-\bfR\tr)^\vee}{2\sin\theta} 
~,
\end{align}
\end{subequations}
%
with $\theta=\cos\inv\left(\frac{\trace(\bfR)-1}{2}\right)$,
and where $\bullet^\vee$, known as the \emph{vee} operator, is the inverse of the \emph{skew} operator $\hatx{\bullet}$. 
Their exact form ($\bfq$ or $\bfR$) is always clear by the context.




%%%%%%%%%%%%%%%%%%%%%%%%%%%%%%%%%%%%%%%%%%%%%%%%%%%%%%%%%%%%

\subsection{Delta definitions}

The deltas $\D\!=\!(\Dp,\Dv,\Dq)$ are defined as the motion (position, velocity, orientation) of a body \wrt a non-rotating frame that is free-falling at the acceleration of gravity $\bfg$. At $t=t_i$, this frame was at position $\bfp_i$ and orientation $\bfq_i$, and moving at velocity $\bfv_i$. From this definition, deltas $\D_{ij}$ from time $t=t_i$ to $t=t_j$ respond to the expression,
%
\begin{align}
\begin{split}
\Dp_{ij} &= \bfq_i^*\od\Big(\bfp_j - \bfp_i - \bfv_i\Dt_{ij} - \frac12\bfg\Dt_{ij}^2\Big) \\
\Dv_{ij} &= \bfq_i^*\od(\bfv_j - \bfv_i - \bfg\Dt_{ij}) \\
\Dq_{ij} &= \bfq_i^*\ot\bfq_j 
\end{split}
\end{align}
%
where $\Dt_{ij} \te t_j - t_i$. Interestingly, the deltas form a group under the composition operator $\D_{ik}\te\D_{ij}\oplus\D_{jk}$ defined by,
%
\begin{align} \label{equ:g_composition}
\begin{split}
\Dp_{ik} 
&= \Dp_{ij} + \Dv_{ij}\Dt_{jk} + \Dq_{ij}\od\Dp_{jk} \\
\Dv_{ik} 
&= \Dv_{ij} + \Dq_{ij}\od\Dv_{jk} \\
\Dq_{ik} 
&= \Dq_{ij}\ot\Dq_{jk} 
\end{split}
\end{align}
%
with identity $\D_1=[(0,0,0),(0,0,0),(1,0,0,0)]$, and inverse $\D_{ji}\te\D_{ij}\inv$ defined by,
%
\begin{align}
\begin{split}
\Dp_{ji} 
&= -\Dq_{ij}^*\od(\Dp_{ij} - \Dv_{ij}\Dt_{ij}) \\
\Dv_{ji} 
&= -\Dq_{ij}^*\od\Dv_{ij} \\
\Dq_{ji} 
&= \Dq_{ij}^*
~.
\end{split}
\end{align}

%\subsection{Body magnitudes $(\bfa,\bw)$}%
%
%\begin{align}
%\begin{split}
%{}^W\bfa &= \bfq_j\od\bfa + \bfg \\
%{}^B\bw &= \bw
%\end{split}
%\end{align}


\subsection{State integration}

We define the world-referenced states of position, velocity, and orientation quaternion, $(\bfp,\bfv,\bfq)$. 
Their time evolution is governed by the kinematic equation,
%
\begin{align}\label{equ:cont_basic}
\dot\bfp &= \bfv \\
\dot\bfv &= \bfq\od\bfa + \bfg \\
\dot\bfq &= \frac12\bfq\ot\bw 
\end{align}
%
where we identify $(\bfa,\bw)$ as the \emph{body magnitudes}, that is, the magnitudes of acceleration and angular rate in the IMU reference frame.
Assuming constant body magnitudes within the interval $\dt\te t_k-t_j$, we have the discrete-time relation
%
\begin{align}\label{equ:g_second_order_integration}
\begin{split}
\bfp_{k} &= \bfp_j + \bfv_j\dt  + \frac12\bfg\dt^2 + \frac12\bfq_j\od\bfa_j\dt^2 \\
\bfv_{k} &= \bfv_j + \bfg\dt + \bfq_j\odot\bfa_j\dt \\
\bfq_{k} &= \bfq_j\ot\Exp(\bw_j\dt/2) 
\end{split}
\end{align}


\subsection{Incremental delta pre-integration}
%
Substituting the delta definitions in the integration equation, we obtain the incremental delta pre-integration,
%
\begin{align}\label{equ:g_second_order_pre-integration}
\begin{split}
\Dp_{ik} 
&= \Dp_{ij} + \Dv_{ij}\dt + \frac12\Dq_{ij}\od\bfa_j\dt^2 \\
\Dv_{ik} 
&= \Dv_{ij} + \Dq_{ij}\od\bfa_j\dt \\
\Dq_{ik} 
&= \Dq_{ij}\ot\Exp(\bw_j\dt) 
\end{split}
\end{align}
%
%which is akin to \eqRef{equ:g_second_order_integration}, that is, the motion equation of a body in an inertial frame.
%
which can be expressed as a delta composition $\D_{ik}=\D_{ij}\oplus\delta_{jk}$ using \eqRef{equ:g_composition}


\subsection{State reconstruction}

It follows from the Delta definitions
%
\begin{align}
\begin{split}
\bfp_j &= \bfp_i + \bfv_i\Dt_{ij} + \frac12\bfg\Dt_{ij}^2 + \bfq_i\od\Dp_{ij} \\
\bfv_j &= \bfv_i + \bfg\Dt_{ij} + \bfq_i\od\Dv_{ij} \\
\bfq_j &= \bfq_i\ot\Dq_{ij}   
\end{split}
\end{align}

\subsection{Delta composition}

We define the composition operator $\D_{ik} = \D_{ij}\oplus\D_{jk}$, and develop it as follows,
%
%\begin{align*}
%\Dp_{ik} 
%&\te \bfq_i^*\od\Big(\bfp_k - \bfp_i - \bfv_i\Dt_{ik} - \frac12\bfg\Dt_{ik}^2\Big) \\
%&= \bfq_i^*\od\Big((\bfp_j + \bfv_j\Dt_{jk} + \frac12\bfg\Dt_{jk}^2 + \bfq_j\od\Dp_{jk}) \\
%&~~~ - \bfp_i - \bfv_i(\Dt_{ij}+\Dt_{jk}) - \frac12\bfg(\Dt_{ij}+\Dt_{jk})^2\Big) \\
%&= \bfq_i^*\od\Big(\bfp_j + \bfv_j\Dt_{jk} + \frac12\bfg\Dt_{jk}^2 + \bfq_j\od\Dp_{jk} \\
%&~~~ - \bfp_i - \bfv_i\Dt_{ij} - \bfv_i\Dt_{jk} - \frac12\bfg\Dt_{ij}^2 - \bfg\Dt_{ij}\Dt_{jk} - \frac12\bfg\Dt_{jk}^2\Big) \\
%&= \bfq_i^*\od\Big((\bfp_j - \bfp_i - \bfv_i\Dt_{ij} - \frac12\bfg\Dt_{ij}^2) + (\bfv_j - \bfv_i - \bfg\Dt_{ij})\Dt_{jk} + \bfq_j\od\Dp_{jk}\Big) \\
%&= \Dp_{ij} + \Dv_{ij}\Dt_{jk} + \Dq_{ij}\od\Dp_{jk} \\
%\Dv_{ik} 
%&\te \bfq_i^*\od(\bfv_k - \bfv_i - \bfg\Dt_{ik}) \\
%&= \bfq_i^*\od(\bfv_j + \bfg\Dt_{jk} + \bfq_j\od\Dv_{jk} - \bfv_i - \bfg(\Dt_{ij}+\Dt_{jk})) \\
%&= \bfq_i^*\od(\bfv_j - \bfv_i - \bfg\Dt_{ij} + \bfq_j\od\Dv_{jk}) \\
%&= \Dv_{ij} + \Dq_{ij}\od\Dv_{jk}\dt \\
%\Dq_{ik} 
%&\te \bfq_i^*\ot\bfq_k \\
%&= \bfq_i^*\ot\bfq_j\ot\Dq_{jk} \\
%&= \Dq_{ij}\ot\Dq_{jk} 
%\end{align*}
%%
%this leads to
%
%\begin{align} \label{equ:g_composition}
%\begin{split}
%\Dp_{ik} 
%&= \Dp_{ij} + \Dv_{ij}\Dt_{jk} + \Dq_{ij}\od\Dp_{jk} \\
%\Dv_{ik} 
%&= \Dv_{ij} + \Dq_{ij}\od\Dv_{jk} \\
%\Dq_{ik} 
%&= \Dq_{ij}\ot\Dq_{jk} 
%\end{split}
%\end{align}
%%
%which is akin to \eqRef{equ:g_second_order_pre-integration}, that is, we can define a small delta $\delta_{jk}$ representing the last IMU step as
%
\begin{align}
\begin{split}
\dpp_{jk} &= \frac12\bfa_j\dt^2 \\
\dv_{jk} &= \bfa_j\dt \\
\dq_{jk} &= \Exp(\bw_j\dt)
\end{split}
\end{align}
%
and then perform the pre-integration with $\D_{ik} = \D_{ij}\oplus\delta_{jk}$, using \eqRef{equ:g_composition} with $\delta_{jk}$ instead of $\D_{jk}$.


%\subsection{Delta substraction}
%
%Just invert the composition above to get  $\D_{jk} = \D_{ik}\ominus\D_{ij}$,
%%
%\begin{align} 
%\begin{split}
%\Dp_{jk} 
%&= \Dq_{ij}^*\od(\Dp_{ik} - \Dp_{ij} - \Dv_{ij}\Dt_{jk}) \\
%\Dv_{jk} 
%&= \Dq_{ij}^*\od(\Dv_{ik} - \Dv_{ij}) \\
%\Dq_{jk} 
%&= \Dq_{ij}^*\ot\Dq_{ik} 
%\end{split}
%\end{align}
%%
%with $\Dt_{jk} = \Dt_{ik}-\Dt_{ij} = t_k-t_j$.

%\subsection{Delta zero (group identity)}
%
%It is noted $\Dzero$ and defined by
%%
%\begin{align*}
%\Dp^0 &= \bf0 \te \begin{bmatrix}
%0&0&0
%\end{bmatrix}\tr \\
%\Dv^0 &= \bf0 \te \begin{bmatrix}
%0&0&0
%\end{bmatrix}\tr \\
%\Dq^0 &= \bf1 \te \begin{bmatrix}
%1&0&0&0
%\end{bmatrix}\tr
%\end{align*}
%%
%We have that, for any index $i$,
%%
%\begin{align*}
%\Dp_{ii} &= \Dp^0 \\
%\Dv_{ii} &= \Dv^0 \\
%\Dq_{ii} &= \Dq^0 
%\end{align*}

%\subsection{Delta negated (group inverse)}
%%
%It is that delta $\D_{ji}$ such that $\D_{ji} \oplus \D_{ij} = \D_{ij} \oplus \D_{ji} = \Dzero$.
%We develop it from $\D_{ij} \oplus \D_{ji} = \Dzero$,
%%
%\begin{align}
%\begin{split}
%\Dp_{ji} 
%&= -\Dq_{ij}^*\od(\Dp_{ij} - \Dv_{ij}\Dt_{ij}) \\
%\Dv_{ji} 
%&= -\Dq_{ij}^*\od\Dv_{ij} \\
%\Dq_{ji} 
%&= \Dq_{ij}^*
%\end{split}
%\end{align}
%%


%\subsection{Minimal delta (in tangent space)}
%
%This returns a delta in minimal form: position and velocity remain the same, and the orientation is provided in the tangent space,
%%
%\begin{align}
%\begin{split}
%\tan{\Dp} &= \Dp \\
%\tan{\Dv} &= \Dp \\
%\tan{\Dq} &= \Log(\Dq) = \bfu\theta 
%\end{split}
%\end{align}
%%



%%%%%%%%%%%%%%%%%%%%%%%%%%%%%%%%%%%%%%%%%%%%%%%%%%%%%%%%%%%%
%\newpage
\subsection{Jacobians of the body magnitudes}

Note: Uncertainties and perturbed magnitudes are respectively marked by the prefix $\partial$ and the tilde\, $\widetilde\cdot$\,, and are represented as follows,
%
\begin{align*}
\widetilde\Dp &= \Dp + \partial\Dp & \partial\Dp &= \widetilde\Dp-\Dp \\
\widetilde\Dv &= \Dv + \partial\Dv & \partial\Dv &= \widetilde\Dv-\Dv \\
\widetilde\Dq &= \Dq \ot \Exp( \partial\Dth ) & \partial\Dth &= \Log(\Dq^*\ot\widetilde\Dq) \\
\widetilde\DR &= \DR  \Exp( \partial\Dth ) & \partial\Dth &= \Log(\DR\tr\widetilde\DR) 
\end{align*}

%We proceed in two steps: creation of the delta covariance from the data covariance, and integration of the delta covariance. For this, we need to develop the Jacobians of data2delta() and deltaPlusDelta().



We consider noisy body magnitudes, with the noise terms added,%\footnote{The noise sign should be negative according to its definition, \eg~$\bfa_m = \bfa+\ab+\bfa_n$, but since noise is zero-mean Gaussian this sign has no real impact, and changing it makes our formulas simpler.}
%
\begin{align*}
\bfa &= \bfa_m - \ab + \bfa_n \\
\bw &= \bw_m - \wb + \bw_n 
~.
\end{align*}
%
We have immediately
%
\begin{align*}
\bfA_{\bfa_m} 
  &= -\bfA_{\bfa_b} = \bfA_{\bfa_n} = \bfI 
& \bfA_{\bw_m} 
  &= ~~\bfA_{\bw_b} = \bfA_{\bw_n} = \bf0 \\
\bfOmega_{\bfa_m} 
  &= ~~\bfOmega_{\bfa_b} = \bfOmega_{\bfa_n} = \bf0 
& \bfOmega_{\bw_m} 
  &= -\bfOmega_{\bw_b} = \bfOmega_{\bw_n} = \bfI 
\end{align*}


\subsection{Jacobians of the current delta}% \wrt the data}
\label{sec:jac_data}

%We consider noisy body magnitudes, with the noise terms added,\footnote{The noise sign should be negative according to its definition, \eg~$\bfa_m = \bfa+\ab+\bfa_n$, but since noise is zero-mean Gaussian this sign has no real impact, and changing it makes our formulas simpler.}
%
%\begin{align*}
%\bfa &= \bfa_m - \ab + \bfa_n \\
%\bw &= \bw_m - \wb + \bw_n 
%\end{align*}
%
%so that derivatives \wrt $\{\bfa_n,\bw_n\}$ coincide with those \wrt $\{\bfa,\bw\}$.
%
The involved operations are the delta creation, $\delta_{jk} \gets (\bfa_j,\bw_j,\dt)$,
%
\begin{align*}
\begin{split}
\dpp_{jk} &= \frac12\bfa_j\dt^2 \\
\dv_{jk} &= \bfa_j\dt \\
\dq_{jk} &= \Exp(\bw_j\dt)
\end{split}
\end{align*}
%
We obtain most of the Jacobians by simple inspection,
%
\begin{align*}
\dP_\bfa &= \frac12\bfI\dt^2 	& \dP_\bw &= \bf0 \\
\dV_\bfa &= \bfI\dt 			& \dV_\bw &= \bf0 \\
\dTH_\bfa &= \bf0 				& \dTH_\bw &= \bfJ_r(\bw\dt)\dt ~~,\textrm{ (see below)} 
\end{align*}
%
We develop $\dTH_\bw$ as follows,
%
\begin{align*}
\widetilde\dq \te \dq\ot\Exp(\partial\dth) 
&= \Exp((\bw+\partial\bw)\dt) \\
\cJr &= \Exp(\bw\dt)\ot\Exp(\bfJ_r(\bw\dt)\partial\bw\dt) \\
\csubst &= \dq \ot \Exp(\bfJ_r(\bw\dt)\partial\bw\dt)
~,
\end{align*}
%
which leads to
%
\begin{align*}
\partial\dth &= \bfJ_r(\bw\dt)\partial\bw\dt
\end{align*}
%
therefore
%
\begin{align}
\dTH_\bw = \dpar{\dth}{\bw} = \bfJ_r(\bw\dt)\dt
\end{align}

%{\color{red}
%%
%\begin{align*}
%%\begin{split}
%\dTH_\bw \te \dpar{\dth}{\dw} 
%&\te \lim_{\dw\to0}\frac{\Log(\Exp(\bw\dt)\tr\Exp((\bw+\dw)\dt))}{\dw} \\
%\cJr&= \lim_{\dw\to0}\frac{\Log(\Exp(\bw\dt)\tr\Exp(\bw\dt)\Exp(\bfJ_r(\bw\dt)\dw\dt))}{\dw} \\
%\ctrans&= \lim_{\dw\to0}\frac{\Log(\Exp(\bfJ_r(\bw\dt)\dw\dt))}{\dw} \\
%\clog&= \lim_{\dw\to0}\frac{\bfJ_r(\bw\dt)\dw\dt}{\dw} \\
%\clim&= \bfJ_r(\bw\dt)\,\dt
%%\end{split}
%\end{align*}
%%
%which is not surprising since this is so close to the definition of $\bfJ_r$. We highlight this point by re-developing the derivative, this time using the chain rule,
%%
%\begin{align*}
%\begin{split}
%\dpar{\dth_{jk}}{\dw_j} 
%&= \dpar{\dth_{jk}}{(\dw_j\dt)}\cdot\dpar{(\dw_j\dt)}{\dw_j} \\
%\cdef&= \lim_{\dw_j\to0}\frac{r(\bw_j\dt+\dw_j\dt)\ominus r(\bw_j\dt)}{(\dw_j\dt)}\cdot\bfI\,\dt \\
%\cJr&= \bfJ_r(\bw_j\dt)\,\dt
%\end{split}
%\end{align*}
%%
%}
%{\color{blue}
%Or otherwise, we just consider $\bth_{ik}=\bw_j\dt$ as the integration in the tangent space, resulting in the trivial Jacobian,
%%
%\begin{align*}
%\dTH_\bw = \bfI\dt
%\end{align*}%
%}%



\subsection{Jacobians of the delta composition}

The involved operations are the delta composition $\D_{ik}=\D_{ij}\oplus\delta_{jk}$ or $\D^+=\D\oplus\delta$,
%
\begin{align*} \label{equ:g_second_composition}
\begin{split}
\Dp_{ik} 
&= \Dp_{ij} + \Dv_{ij}\dt + \Dq_{ij}\od\dpp_{jk} \\
\Dv_{ik} 
&= \Dv_{ij} + \Dq_{ij}\od\dv_{jk} \\
\Dq_{ik} 
&= \Dq_{ij}\ot\dq_{jk} 
\end{split}
\end{align*}

\subsubsection{Jacobians \wrt the first delta, $\D_{ij}$}
\label{sec:jac_first_delta}

Again by simple inspection we can obtain a number of Jacobians,
%
\begin{align*}
\DP^+_\Dp &= \bfI  & \DP^+_\Dv &= \bfI\dt & \DP^+_\Dth &= - \DR_{ij}  \hatx{\dpp_{jk}}  ~~,\textrm{ (see below)}\\% \bfJ_r(\Dth_{ij}) \\
\DV^+_\Dp &= \bf0  & \DV^+_\Dv &= \bfI & \DV^+_\Dth &= - \DR_{ij}  \hatx{\dv_{jk}}  ~~,\textrm{ (see below)}\\% \bfJ_r(\Dth_{ij}) \\
\DTH^+_\Dp &= \bf0  & \DTH^+_\Dv &= \bf0 & \DTH^+_\Dth &= \dR_{jk}\tr %\,\bfJ_r(\Dth_{ij})  
~~,\textrm{ (see below)}
\end{align*}
%
We develop $\DP_\Dth$ as follows
%
\begin{align*}
\widetilde\Dp_{ik} \te \Dp_{ik}+\partial\Dp_{ik} 
&= \Dp_{ij} + \Dv_{ij}\dt + \widetilde\DR_{ij}\dpp_{jk} \\
&= \Dp_{ij} + \Dv_{ij}\dt + \DR_{ij}\Exp(\partial\Dth_{ij})\dpp_{jk} \\
\csmall
&= \Dp_{ij} + \Dv_{ij}\dt + \DR_{ij}(\bfI+\hatx{\partial\Dth_{ij}})\dpp_{jk} \\
&= \underbrace{\Dp_{ij} + \Dv_{ij}\dt + \DR_{ij}\dpp_{jk}} + \DR_{ij}\hatx{\partial\Dth_{ij}}\dpp_{jk} \\
\com{subst, cross}
&= ~~~~~~~~~~~~~~~\Dp_{ik} ~~~~~~~~~~~~~  - \DR_{ij}\hatx{\dpp_{jk}}\partial\Dth_{ij} 
\end{align*}
%
therefore,
%
\begin{align}
\DP^+_\Dth \te \dpar{\Dp_{ik}}{\Dth_{ij}} = - \DR_{ij}\hatx{\dpp_{jk}}
\end{align}
%
Similarly for the velocity,
%
\begin{align}
\DV^+_\Dth \te \dpar{\Dv_{ik}}{\Dth_{ij}} = - \DR_{ij}\hatx{\dv_{jk}}
\end{align}
%
Finally for the orientation,
%
\begin{align*}
\widetilde\DR_{ik} \te \DR_{ik}\Exp(\partial\Dth_{ik}) 
&= \widetilde\DR_{ij}\dR_{jk} \\
&= \DR_{ij}\Exp(\partial\Dth_{ij})\dR_{jk}  \\
\cswap
&= \DR_{ij}\dR_{jk}\Exp(\dR_{jk}\tr\partial\Dth_{ij})  \\
&= \DR_{ik}\Exp(\dR_{jk}\tr\partial\Dth_{ij})  
\end{align*}
%
which leads to
%
\begin{align*}
\Dth_{ik} &= \dR_{jk}\tr\partial\Dth_{ij}
\end{align*}
%
therefore,
%
\begin{align}
\DTH^+_\Dth \te \dpar{\Dth_{ik}}{\Dth_{ij}} = \dR_{jk}\tr
\end{align}




%{\color{red}
%\bigskip
%We develop $\DP_\Dth$ as follows (see Example 1 in \appRef{sec:DosDonts})
%%
%\begin{align*}
%\DP_\Dth = \dpar{\Dp_{ik}}{\Dth_{ij}} 
%= \dpar{(\Dq_{ij}\od\dpp_{jk})}{\Dth_{ij}}
%&= - \DR_{ij}  \hatx{\dpp_{jk}}  \bfJ_r(\Dth_{ij})
%\end{align*}
%%
%Similarly for $\DV_\Dth$,
%%
%\begin{align*}
%\DV_\Dth = \dpar{\Dv_{ik}}{\Dth_{ij}} 
%= \dpar{(\Dq_{ij}\od\dv_{jk})}{\Dth_{ij}}
%&= - \DR_{ij}  \hatx{\dv_{jk}}  \bfJ_r(\Dth_{ij})
%\end{align*}
%%
%And finally for $\DTH_\Dth$\,, 
%%
%%{\color{red}
%%This first development is wrong: $Jr(\theta_{ij}) \ne I+\frac12[\theta_{ij}]_\times$, because $\theta_{ij}$ is not small!
%%%
%%\begin{align*}
%%\DTH_\Dth = \dpar{\Dth_{ik}}{\Dth_{ij}} 
%%&= \dpar{\Log(\Exp(\Dth_{ij})\Exp(\dth_{jk})}{\Dth_{ij}} \\
%%&= \dpar{(\Dth_{ij} + \bfJ_r(\Dth_{ij})\inv\dth_{jk})}{\Dth_{ij}} \\
%%&= \dpar{(\Dth_{ij} + (\bfI+\frac12\hatx{\Dth_{ij}})\dth_{jk})}{\Dth_{ij}} \\
%%&= \dpar{(\Dth_{ij} -\frac12\hatx{\dth_{jk}}\Dth_{ij})}{\Dth_{ij}} \\
%%&= \bfI - \frac12\hatx{\dth_{jk}} 
%%\end{align*}
%%%
%%or otherwise
%%} %% \color(red)
%%
%\begin{align*}
%\DTH_\Dth 
%&= \dpar{\Dth_{ik}}{\Dth_{ij}} \\
%\cdef
%&\te \lim_{\partial\Dth_{ij}\to0} \frac{\Log\Big(\big(\Exp(\Dth_{ij})\Exp(\dth_{jk})\big)\tr\big(\Exp(\Dth_{ij}+\partial\Dth_{ij})\Exp(\dth_{jk})\big)\Big)}{\partial\Dth_{ij}} \\
%\com{$\cdot\tr$,Jr}
%&= \lim_{\partial\Dth_{ij}\to0} \frac{\Log(\Exp(\dth_{jk})\tr\cancel{\Exp(\Dth_{ij})}\tr\cancel{\Exp(\Dth_{ij})}\Exp(\bfJ_r(\Dth_{ij})\partial\Dth_{ij})\Exp(\dth_{jk}))}{\partial\Dth_{ij}} \\
%\ccancel
%&= \lim_{\partial\Dth_{ij}\to0} \frac{\Log(\Exp(\dth_{jk})\tr\Exp(\bfJ_r(\Dth_{ij})\partial\Dth_{ij})\Exp(\dth_{jk}))}{\partial\Dth_{ij}} \\
%\cswap
%&= \lim_{\partial\Dth_{ij}\to0} \frac{\Log(\Exp(\Exp(\dth_{jk})\tr\bfJ_r(\Dth_{ij})\partial\Dth_{ij}))}{\partial\Dth_{ij}} \\
%\clog
%&= \lim_{\partial\Dth_{ij}\to0} \frac{\Exp(\dth_{jk})\tr\bfJ_r(\Dth_{ij})\partial\Dth_{ij}}{\partial\Dth_{ij}} \\
%\clim
%&= \Exp(\dth_{jk})\tr\bfJ_r(\Dth_{ij}) \\
%\csubst
%&= \dR_{jk}\tr\,\bfJ_r(\Dth_{ij}) 
%\end{align*}
%
%} % red

\subsubsection{Jacobians \wrt the second delta, $\delta_{jk}$}
\label{sec:jac_second_delta}

Again by simple inspection we can obtain a number of Jacobians,
%
\begin{align*}
\DP^+_\dpp &= \DR_{ij} & \DP^+_\dv &= \bf0     & \DP^+_\dth &= \bf0 \\
\DV^+_\dpp &= \bf0     & \DV^+_\dv &= \DR_{ij} & \DV^+_\dth &= \bf0 \\
\DTH^+_\dpp &= \bf0    & \DTH^+_\dv &= \bf0    & \DTH^+_\dth &= \bfI %\bfJ_r(\dth_{jk})  
~~,\textrm{ (see below)}
\end{align*}
%
We develop $\DTH_\dth$ as follows,
%
\begin{align*}
\widetilde\DR_{ik} = \DR_{ik}\Exp(\partial\Dth_{ik}) 
&= \DR_{ij}\widetilde\dR_{jk} \\
&= \DR_{ij}\dR_{jk}\Exp(\partial\dth_{jk}) \\ 
&= \DR_{ik}\Exp(\partial\dth_{jk}) 
\end{align*}
%
which leads to
%
\begin{align*}
\partial\Dth_{ik} &= \partial\dth_{jk}
\end{align*}
%
therefore,
%
\begin{align}
\DTH^+_\dth \te \dpar{\Dth_{ik}}{\dth_{jk}} = \bfI
\end{align}





\subsection{Jacobians of the direct integration}

By chaining the three steps of computation of the body magnitudes, delta creation and delta composition, we encounter the Jacobians of the direct integration. The operations of direct integration are 
%
\begin{align*}
\Dp_{ik} 
&= \Dp_{ij} + \Dv_{ij}\dt + \frac12\Dq_{ij}\od(\bfa_{m,j}-\bfa_b)\dt^2 \\
\Dv_{ik} 
&= \Dv_{ij} + \Dq_{ij}\od(\bfa_{m,j}-\bfa_b)\dt \\
\Dq_{ik} 
&= \Dq_{ij}\ot\Exp((\bw_{m,j}-\bw_b, \dt)\dt) 
\end{align*}
%
which boils down to
%
\begin{align*}
\D_{ik} &= \D_{ij} \oplus \delta( (\bfa_{m,j}-\bfa_b), (\bw_{m,j}-\bw_b, \dt) )
\end{align*}
%

\subsubsection{Jacobians \wrt the previous delta $\D_{ij}$}

These are the ones detailed before, that we rewrite here,
%
\begin{align*}
\DP^+_\Dp &= \bfI  & \DP^+_\Dv &= \bfI\dt & \DP^+_\Dth &= - \DR_{ij}  \hatx{\dpp_{jk}} \\
\DV^+_\Dp &= \bf0  & \DV^+_\Dv &= \bfI & \DV^+_\Dth &= - \DR_{ij}  \hatx{\dv_{jk}}  \\
\DTH^+_\Dp &= \bf0  & \DTH^+_\Dv &= \bf0 & \DTH^+_\Dth &= \dR_{jk}\tr 
\end{align*}
\subsubsection{Jacobians \wrt noise}

{\color{red}
The Jacobians \wrt the body magnitudes $(\bfa,\bw)$ match those \wrt noise, and are obtained through the chain rule,
%
\begin{align*}
%
\DP^+_\bfa  &= \DP^+_\DP\DP_\bfa + \DP^+_\dP\dP_\bfa 
& \DP^+_\bw &= \DP^+_\DP\DP_\bw + \DP^+_\dP\dP_\bw \\
%
\DV^+_\bfa  &= \DV^+_\DV\DV_\bfa + \DV^+_\dV \, \dV_\bfa
& \DV^+_\bw &= \DV_\dv\,\dV_\bw = \bf0 \\
%
\DTH^+_\bfa  &= \bf0 
& \DTH^+_\bw &= \DTH_\dth \, \dTH_\bw = \bfJ_r(\bw_j\dt)\dt 
%
\end{align*}
%
which results in
%
\begin{align*}
%
\DP^+_\bfa  &= \DP^+_\DP\DP_\bfa + \DP^+_\dP\dP_\bfa = \frac12\DR_{ij}\dt^2 
& \DP^+_\bw &= \DP_\dpp\,\dP_\bw    = \bf0 \\
%
\DV^+_\bfa  &= \DV_\dv \, \dV_\bfa = \DR_{ij}\dt
& \DV^+_\bw &= \DV_\dv\,\dV_\bw = \bf0 \\
%
\DTH^+_\bfa  &= \bf0 
& \DTH^+_\bw &= \DTH_\dth \, \dTH_\bw = \bfJ_r(\bw_j\dt)\dt 
%
\end{align*}
%
}

\subsubsection{Jacobians \wrt the biases}

Also by the chain rule with $\bfA_{\bfa_b}=-\bfI$ and $\bfOmega_{\bw_b}=-\bfI$, we just have to swap some signs to the above as in \eg~$\DP_{\bfa_b}  = \DP_{\bfa_b}\bfA_{\bfa_b} = - \DP_{\bfa}$, to get
%
\begin{align*}
%
\DP_{\bfa_b}  &= - \frac12\DR_{ij}\dt^2 
& \DP_{\bw_b} &=  \bf0 \\
%
\DV_{\bfa_b}  &= - \DR_{ij}\dt
& \DV_{\bw_b} &=  \bf0 \\
%
\DTH_{\bfa_b}  &= \bf0 
& \DTH_{\bw_b} &= - \bfJ_r(\bw_j\dt)\dt 
%
\end{align*}




%%%%%%%%%%%%%%%%%%%%%%%%%%%%%%%%%%%%%%%%%%%%%%%%%%%%%%%%%%%%

\section{Summary -- 2nd order, with gravity}
\label{sec:summary-gravity-2nd}

\subsection{Delta definitions}

Motion \wrt a non-rotating, free-falling reference frame
%
\begin{align}
\begin{split}
\Dp_{ij} &= \bfq_i^*\od\Big(\bfp_j - \bfp_i - \bfv_i\Dt_{ij} - \frac12\bfg\Dt_{ij}^2\Big) \\
\Dv_{ij} &= \bfq_i^*\od(\bfv_j - \bfv_i - \bfg\Dt_{ij}) \\
\Dq_{ij} &= \bfq_i^*\ot\bfq_j 
\end{split}
\end{align}
%
We may write this as $\D_{ij}=\bfx_j\ominus\bfx_i$

\subsection{Delta pre-integration}

\subsubsection{Body magnitudes from measurements with bias}

\begin{align}
\begin{split}
\bfa &= \bfa_m - \ab \\
\bw &= \bw_m - \wb
\end{split}
\end{align}
%
with Jacobians
%
\begin{align*}
\bfA_{\bfa_m} 
  &= -\bfA_{\bfa_b} = \bfA_{\bfa_n} = \bfI 
& \bfA_{\bw_m} 
  &= ~~\bfA_{\bw_b} = \bfA_{\bw_n} = \bf0 \\
\bfOmega_{\bfa_m} 
  &= ~~\bfOmega_{\bfa_b} = \bfOmega_{\bfa_n} = \bf0 
& \bfOmega_{\bw_m} 
  &= -\bfOmega_{\bw_b} = \bfOmega_{\bw_n} = \bfI 
\end{align*}


\subsubsection{Direct method}

\begin{align} \label{equ:integration}
\begin{split}
\Dp_{ik} 
&= \Dp_{ij} + \Dv_{ij}\dt + \frac12\Dq_{ij}\od\bfa_j\dt^2 \\
\Dv_{ik} 
&= \Dv_{ij} + \Dq_{ij}\od\bfa_j\dt \\
\Dq_{ik} 
&= \Dq_{ij}\ot\Exp(\bw_j\dt) 
\end{split}
\end{align}
%
with Jacobian $\D_\D$ formed by the blocks,
%
\begin{align*}
\DP_\Dp &= \bfI  & \DP_\Dv &= \bfI\dt & \DP_\Dth &= - \frac12\DR_{ij}  \hatx{\bfa_j}\dt^2 \\
\DV_\Dp &= \bf0  & \DV_\Dv &= \bfI & \DV_\Dth &= - \DR_{ij}  \hatx{\bfa_j}\dt \\
\DTH_\Dp &= \bf0  & \DTH_\Dv &= \bf0 & \DTH_\Dth &= \dR_{jk}\tr
\end{align*}
%
and $\D_n$ formed by
%
\begin{align*}
\DP_\bfa  &= \DP_\dpp \, \dP_\bfa = \frac12\DR_{ij}\dt^2 
& \DP_\bw &= \DP_\dpp\,\dP_\bw    = \bf0 \\
\DV_\bfa  &= \DV_\dv \, \dV_\bfa = \DR_{ij}\dt
& \DV_\bw &= \DV_\dv\,\dV_\bw = \bf0 \\
\DTH_\bfa  &= \bf0 
& \DTH_\bw &= \DTH_\dth \, \dTH_\bw = \bfJ_r(\bw_j\dt)\dt 
\end{align*}
%
The covariances matrix is integrated with
%
\begin{align}
\Sigma_\D|_{ik} = \D_\D\, \Sigma_\D|_{ij}\, \D_\D\tr + \D_n\, \Sigma_n \,\D_n\tr
\end{align}



\subsubsection{Method with a temporary delta $\delta$}

\paragraph{Delta creation}
We define a $\delta$ from the body magnitudes, $\delta \gets (\bfa, \bw, \dt)$,
%
\begin{subequations}
\begin{align} \label{equ:delta}
\begin{split}
\dpp_{jk} &= \frac12\bfa_j\dt^2 \\
\dv_{jk} &= \bfa_j\dt \\
\dq_{jk} &= \Exp(\bw_j\dt)
\end{split}
\end{align}
%
with $k=j+1$. The Jacobian $\D_m$ is formed by the blocks,
%
\begin{align*}
\dP_\bfa &= \frac12\bfI\dt^2 	& \dP_\bw &= 0 \\
\dV_\bfa &= \bfI\dt 			& \dV_\bw &= 0 \\
\dTH_\bfa &= 0 					& \dTH_\bw &= \bfJ_r(\bw_j\dt)\dt  
\end{align*}
%
The covariance of the delta is created with
%
\begin{align*}
\Sigma_\delta = \D_m \,\Sigma_m \,\D_m\tr
\end{align*}


\paragraph{Delta composition}
Then,
%
$\D \gets \D \oplus \delta$,
%
\begin{align} \label{equ:composition_delta}
\begin{split}
\Dp_{ik} 
&= \Dp_{ij} + \Dv_{ij}\dt + \Dq_{ij}\od\dpp_{jk} \\
\Dv_{ik} 
&= \Dv_{ij} + \Dq_{ij}\od\dv_{jk} \\
\Dq_{ik} 
&= \Dq_{ij}\ot\dq_{jk} 
\end{split}
\end{align}
\end{subequations}
%
The Jacobian $\D_\D$ is formed by the blocks
%
\begin{align*}
\DP_\Dp &= \bfI  & \DP_\Dv &= \bfI\dt & \DP_\Dth &= - \DR_{ij}  \hatx{\dpp_{jk}} \\
\DV_\Dp &= \bf0  & \DV_\Dv &= \bfI & \DV_\Dth &= - \DR_{ij}  \hatx{\dv_{jk}} \\
\DTH_\Dp &= \bf0  & \DTH_\Dv &= \bf0 & \DTH_\Dth &= \dR_{jk}\tr 
\end{align*}
%
and $\D_\delta$ by the blocks
%
\begin{align*}
\DP_\dpp &= \DR_{ij} & \DP_\dv &= \bf0     & \DP_\dth &= \bf0 \\
\DV_\dpp &= \bf0     & \DV_\dv &= \DR_{ij} & \DV_\dth &= \bf0 \\
\DTH_\dpp &= \bf0    & \DTH_\dv &= \bf0    & \DTH_\dth &= \bfI
\end{align*}
%
The covariance of the Delta is integrated according to
%
\begin{align*}
\Sigma_\D|_{ik} = \D_\D\,\Sigma_\D|_{ij}\,\D_\D + \D_\delta\,\Sigma_\delta\,\D_\delta
\end{align*}


\subsection{Delta algebra}

\subsubsection{Addition (group operator)}

\begin{align} \label{equ:composition_Delta}
\begin{split}
\Dp_{ik} 
&= \Dp_{ij} + \Dv_{ij}\Dt_{jk} + \Dq_{ij}\od\Dp_{jk} \\
\Dv_{ik} 
&= \Dv_{ij} + \Dq_{ij}\od\Dv_{jk} \\
\Dq_{ik} 
&= \Dq_{ij}\ot\Dq_{jk} 
\end{split}
\end{align}
%
which is akin to \eqRef{equ:composition_delta}, that is, the same composition applies for big deltas $\D$ and for small deltas $\delta$.


\subsubsection{Substraction}

\begin{align} 
\begin{split}
\Dp_{jk} 
&= \Dq_{ij}^*\od(\Dp_{ik} - \Dp_{ij} - \Dv_{ij}\Dt_{jk}) \\
\Dv_{jk} 
&= \Dq_{ij}^*\od(\Dv_{ik} - \Dv_{ij}) \\
\Dq_{jk} 
&= \Dq_{ij}^*\ot\Dq_{ik} 
\end{split}
\end{align}

\subsubsection{Delta zero (group identity)}
%
\begin{align*}
\Dp^0 &= \bf0 \te \begin{bmatrix}
0&0&0
\end{bmatrix}\tr \\
\Dv^0 &= \bf0 \te \begin{bmatrix}
0&0&0
\end{bmatrix}\tr \\
\Dq^0 &= \bf1 \te \begin{bmatrix}
1&0&0&0
\end{bmatrix}\tr
\end{align*}

\subsubsection{Negative Delta (group inverse)}

\begin{align}
\begin{split}
\Dp_{ji} 
&= -\Dq_{ij}^*\od(\Dp_{ij} - \Dv_{ij}\Dt_{ij}) \\
\Dv_{ji} 
&= -\Dq_{ij}^*\od\Dv_{ij} \\
\Dq_{ji} 
&= \Dq_{ij}^*
\end{split}
\end{align}

\subsubsection{Minimal delta (in tangent space)}

%
\begin{align}
\begin{split}
\tan{\Dp} &= \Dp \\
\tan{\Dv} &= \Dv \\
\tan{\Dq} &= \Log(\Dq) = \bfu\theta 
\end{split}
\end{align}
%
with
%
\begin{align*}
\Log(\bfq) = \bfu\theta &= 2\,\qv\frac{\arctan({\norm{\qv},q_w})}{\norm{\qv}} \\
&\approx 2\,\frac{\qv}{q_w} \left(1 - \frac{\norm{\qv}^2}{3q_w^2}\right)
\end{align*}


\subsection{Delta correction with new bias}

\subsubsection{Before integration}

Initialize the Jacobians
%
\begin{equation}
\DP_{\ab}|_{ii} = \DV_{\ab}|_{ii} = \DP_{\wb}|_{ii} = \DV_{\wb}|_{ii} = \DTH_{\bfw_b}|_{ii} = {\bf0}_{3\times3}
\end{equation}
%

\subsubsection{During integration}

Update the Jacobians,
%
\begin{align} 
\begin{split}
\DP_{\ab} |_{ik}
&= \DP_{\ab}|_{ij} + \DV_{\ab}|_{ij}\dt -\frac12\DR_{ij}\dt^2 \\
\DV_{\ab}|_{ik} 
&= \DV_{\ab}|_{ij} - \DR_{ij}\dt \\ 
\DP_{\wb} |_{ik}
&= \DP_{\wb}|_{ij} + \DV_{\wb}|_{ij}\dt -\frac12\DR_{ij}\hatx{\bfa_{mj}-\ab}\DTH_{\bfw_b}|_{ij}\,\dt^2 \\
\DV_{\wb}|_{ik} 
&= \DV_{\wb}|_{ij} - \DR_{ij}\hatx{\bfa_{mj}-\ab}\DTH_{\bfw_b}|_{ij}\,\dt \\
\DTH_{\bfw_b}|_{ik} 
&= \ol\dR_{jk}\tr\,\DTH_{\wb}|_{ij} - \bfJ_r((\bw_j-\ol\wb)\,\dt)\,\dt
\end{split}
\end{align}
%




\subsubsection{When using the Deltas}

Correct the Deltas according to variations in the bias estimates,
%
\begin{align}
\begin{split}
\Dp &= \ol\Dp + \DP_{\ab}\D\ab  + \DP_{\wb}\D\wb  \\ 
\Dv &= \ol\Dv + \DV_{\ab}\D\ab  + \DV_{\wb}\D\wb  \\
\Dq &= \ol\Dq\ot\Exp(\DTH_{\bfw_b}\D\wb)
\end{split}
\end{align}

\subsection{State reconstruction}

At any time $j$ we can recover the state estimate $\bfx_j$ given the state estimate $\bfx_i$ and the (corrected) delta $\D_{ij}$,
%
\begin{align} \label{equ:reconstruction}
\begin{split}
\bfp_j &= \bfp_i + \bfv_i\Dt_{ij} + \frac12\bfg\Dt_{ij}^2 + \bfq_i\od\Dp_{ij} \\
\bfv_j &= \bfv_i + \bfg\Dt_{ij} + \bfq_i\od\Dv_{ij} \\
\bfq_j &= \bfq_i\ot\Dq_{ij}   
\end{split}
\end{align}

\subsection{Residual}

We compare the corrected delta given the bias estimate $\hat\bfx_b$, against the predicted delta given two motion estimates $(\hat\bfx_i,\hat\bfx_j)$, and express it in minimal space,
%
\begin{align*}
\bfr_{ij}(\hat\bfx_i,\hat\bfx_j,\hat\bfx_{b,i}) = \tan\left( \Big(\ol\D_{ij} \oplus \dpar{\D_{ij}}{\bfx_b} (\hat\bfx_{b,i} - \ol\bfx_{b,i})\Big) \ominus \D_{ij}(\hat\bfx_i , \hat\bfx_j)  \right)
\end{align*}





%%%%%%%%%%%%%%%%%%%%%%%%%%%%%%%%%%%%%%%%%%%%%%%%%%%%%%%%%%%%
\newpage
\appendix

\section{Definition of the derivatives }

\subsection{The additive and subtractive operators in $SO(3)$}

In vector spaces $\bbR^n$, the addition and subtraction operations are performed with the regular sum `$+$' and minus `$-$' operations.
In $SO(3)$ this is not possible, but equivalent operators are needed for establishing a proper calculus corpus. 

We thus define the plus and minus operators, $\oplus,\ominus$, between elements $\sR\in SO(3)$, and elements $\bth\in\bbR^3$ of the tangent space at $\sR$, as follows.

\paragraph{The plus operator.}
The `plus' operator $\oplus:SO(3)\times\bbR^3\to SO(3)$ produces an element $\sS$ of $SO(3)$ which is the result of composing a reference element $\sR$ of $SO(3)$ with a (often small) rotation specified by a vector of $\bth\in\bbR^3$ in the vector space tangent to the reference element $\sR$,
%
\begin{align}
\sS = \sR\oplus \bth &\te \sR\circ\Exp(\bth) && \sR,\sS\in SO(3),~ \bth\in\bbR^3 
\end{align}
%
Notice that this operator may be defined for any representation of $SO(3)$. In particular, for the quaternion and rotation matrix we have,
%
\begin{align}
\bfq_\sS &= \,\bfq_\sR\oplus\bth = \bfq_\sR\ot\Exp(\bth) \\
\bfR_\cS &= \bfR_\sR\oplus \bth = \bfR_\sR\Exp(\bth) 
\end{align}

\paragraph{The minus operator.}
The `minus' operator $\ominus:SO(3)\times SO(3)\to\bbR^3$ is the inverse of the above. It returns the vectorial angular difference $\bth\in\bbR^3$ between two elements of $SO(3)$. This difference is expressed in the  vector space tangent to the reference element $\sR$, 
%
\begin{align}
\bth=\sS\ominus \sR
&\te \Log(\sR\inv \circ \sS)     && \sR,\sS\in SO(3),~ \bth\in\bbR^3  
\end{align}
%
which for the quaternion and rotation matrix reads,
%
\begin{align}
\bth &= \,\,\bfq_\sS\ominus\bfq_\sR\, = \Log(\bfq_\sR^*\ot\bfq_\sS)                      \\
\bth &= \bfR_\sS\ominus\bfR_\sR = \Log(\bfR_\sR\tr\,\bfR_\sS)                         
\end{align}

\bigskip
In both cases, notice that even though the vector difference $\bftheta$ is typically supposed to be small, the definitions above hold for any value of $\bftheta$ (up to the first coverage of the $SO(3)$ manifold, that is, for angles $\theta<\pi$).

\subsection{The four possible derivative definitions}



\subsubsection{Functions from vector space to vector space}

The scalar and vector cases follow the classical definition of the derivative: given a function $f:\bbR^m\to\bbR^n$, we use $\{+,-\}$ to define the derivative as
%
\begin{align}
\dpar{f(\bfx)}{\bfx} &\te \lim_{\delta\bfx\to0}\frac{f(\bfx+\delta\bfx)-f(\bfx)}{\delta\bfx} &&\in \bbR^{n\times m} \label{equ:derivative_vector}
\end{align}
%
Euler integration produces linear expressions of the form
%
\begin{align*}
f(\bfx+\Delta\bfx) &\approx f(\bfx) + \dpar{f(\bfx)}{\bfx}\Delta\bfx
& \in \bbR^n
\end{align*}

\subsubsection{Functions from $SO(3)$ to $SO(3)$}

Given a function $f:SO(3) \to SO(3)$ with $\sR\in SO(3)$ and a local, small angular variation $\bth\in\bbR^3$, we use $\{\oplus,\ominus\}$ to define the derivative as
%
\begin{align}
\dpar{f(\sR)}{\bth} 
&\te \lim_{\delta\bth\to0}\frac{f(\sR\oplus\delta\bth)\ominus f(\sR)}{\delta\bth}  && \in \bbR^{3\times 3}\\
&= \lim_{\delta\bth\to0}\frac{\Log\big(f\inv(\sR)\,f(\sR\Exp(\delta\bth))\big)}{\delta\bth} \label{equ:derivative_SO3}
\end{align}
%
Euler integration produces expressions of the form,
%
\begin{align*}
f(\sR\oplus\Delta\bth) &\approx f(\sR)\,\oplus\,\dpar{f(\sR)}{\bth}\,\Delta\bth
 \te f(\sR)\Exp\left(\dpar{f(\sR)}{\bth}\Delta\bth\right)
 & \in SO(3)
\end{align*}




\subsubsection{Functions from vector space to $SO(3)$}

For the case of a function $f:\bbR^m\to SO(3)$, we use `+' for the vector perturbations, and `$\ominus$' for the $SO(3)$ difference,
%
\begin{align}
\dpar{f(\bfx)}{\bfx} &\te \lim_{\delta\bfx\to0} \frac{ f(\bfx+\delta\bfx)\ominus f(\bfx)}{\delta\bfx} && \in \bbR^{3\times m} \label{equ:dif_RtoSO3}\\
&= \lim_{\delta\bfx\to0} \frac{\Log(f\inv(\bfx) f(\bfx+\delta\bfx))}{\delta\bfx}
\end{align}
%
Euler integration produces expressions of the form,
%
\begin{align*}
f(\bfx+\Delta\bfx) &\approx f(\bfx)\,\oplus\,\dpar{f(\bfx)}{\bfx}\,\Delta\bfx
 \te f(\bfx)\,\Exp\left(\dpar{f(\bfx)}{\bfx}\Delta\bfx\right)
 & \in SO(3)
\end{align*}

\subsubsection{Functions from $SO(3)$ to vector space}

For the case of a function $f: SO(3)\to\bbR^n$, we use `$\oplus$' for the $SO(3)$ perturbations, and `$-$' for the vector difference,
%
\begin{align}
\dpar{f(\sR)}{\bth} &\te \lim_{\delta\bth\to0} \frac{f(\sR\oplus\delta\bth) - f(\sR)}{\delta\bth} && \in \bbR^{n\times 3} \label{equ:jacobian_SO3_Rn}\\
&= \lim_{\delta\bth\to0} \frac{f(\sR\Exp(\delta\bth)) - f(\sR)}{\delta\bth}
\end{align}
%
Euler integration produces expressions of the form,
%
\begin{align*}
f(\sR\oplus\delta\bth) &\approx f(\sR)+\dpar{f(\sR)}{\bth}\,\Delta\bth
 \te f(\sR)+\Exp\left(\dpar{f(\sR)}{\bth}\Delta\bth\right)
 & \in SO(3)
\end{align*}


\subsection{Right Jacobian of $SO(3)$ }

We define the right Jacobian of $SO(3)$ as, 
%
\begin{align}
\bfJ_r(\bth) &\te \dpar{\Exp(\bth)}{\bth} 
\end{align}
%
Since the exponential $\Exp()$ is an application $\bbR^3\to SO(3)$,
we implement this derivative using \eqRef{equ:dif_RtoSO3},
%
\begin{align}
\bfJ_r(\bth) &= \lim_{\dth\to0}\frac{\Exp(\bth+\dth)\ominus\Exp(\bth)}{\dth} \\
 &= \lim_{\dth\to0}\frac{\Log(\Exp(\bth)\tr\Exp(\bth+\dth))}{\dth} && \textrm{if using $\bfR$} \\
 &= \lim_{\dth\to0}\frac{\Log(\Exp(\bth)^*\ot\Exp(\bth+\dth))}{\dth} && \textrm{if using $\bfq$} 
 ~.
\end{align}
%
It has the properties, for any $\bth$ and small $\dth$,
%
\begin{align}
\Exp(\bth+\dth) &\approx \Exp(\bth)\Exp(\bfJ_r(\bth)\dth) \\
\Exp(\bth)\Exp(\dth) &\approx \Exp(\bth+\bfJ_r\inv(\bth)\,\dth) \\
\Log(\Exp(\bth)\Exp(\dth)) &\approx \bth+\bfJ_r\inv(\bth)\,\dth 
\end{align}

The right Jacobian and its inverse can be computed in closed form with
%
\begin{align}
\bfJ_r(\bth) &= \bfI - \frac{1-\cos\nth}{\nth^2}\hatx{\bth} + \frac{\nth-\sin\nth}{\nth^3}\hatx{\bth}^2 \\
\bfJ_r\inv(\bth) &= \bfI + \frac12\hatx{\bth} + \left(\frac1{\nth^2} - \frac{1+\cos\nth}{2\nth\sin\nth}\right)\hatx{\bth}^2
\end{align}






%%%%%%%%%%%%%%%%%%%%%%%%%%%%%%%%%%%%%%%%%%%%%%%%%%%%%%%%%%%%%%%%%%%%%%%%%%%
\newpage
\section{Rules, do's and don'ts for Jacobians}
\label{sec:DosDonts}

We provide a collection of rules which come very handy to develop Jacobians. They come organized under helper \com{keys}\!\!\!\!, which we use to refer to each of these properties in our developments.

\subsection{Useful properties: Do's}

\paragraph{\cchain : Chain rule}

\begin{align}
\dpar{\bfz}{\bfx} = \dpar{\bfz}{\bfy}\cdot\dpar{\bfy}{\bfx}
\end{align}

\paragraph{\ccross : Cross product and skew-symmetric matrix}

\begin{align}
\hatx{\bfa}\bfb &= \bfa\times\bfb \\
\hatx{\bfa}\bfb &= -\hatx{\bfb}\bfa \\
\bfR\tr\hatx{\bfR\bfa}\bfR &= \hatx{\bfa} \\
\hatx{\bfR\bfa} &= \bfR\hatx{\bfa}\bfR\tr 
\end{align}

\paragraph{\cJr : Right Jacobian of $SO(3)$ }

It has the properties, for any $\bth$ and small $\dth$,
%
\begin{align}
\Exp(\bth+\dth) &\approx \Exp(\bth)\Exp(\bfJ_r(\bth)\dth) \\
\Exp(\bth)\Exp(\dth) &\approx \Exp(\bth+\bfJ_r\inv(\bth)\,\dth) \\
\Log(\Exp(\bth)\Exp(\dth)) &\approx \bth+\bfJ_r\inv(\bth)\dth %\\
\end{align}
%






\paragraph{\csmall : Small angle approximations}

Let $\dth$ be a small angle vector. Then,
%
\begin{align}
\Exp(\dth) &\approx \bfI + \hatx{\dth} \\
\Exp(\dth)\tr &\approx \bfI - \hatx{\dth} \\
\Exp(\dth_1)\Exp(\dth_2) &\approx \Exp(\dth_1+\dth_2) \\
\textstyle\prod_i \Exp(\dth_i) &\approx \Exp\!\big(\textstyle\sum_i\dth_i\big) \\
\bfJ_r(\dth) &\approx \bfI - \frac12\hatx{\dth} \\
\bfJ_r\inv(\dth) &\approx \bfI + \frac12\hatx{\dth} 
\end{align}
%
Example: we often use $\Exp(\bfJ_r(\bth)\dth)\approx \bfI + \hatx{\bfJ_r(\bth)\dth}$.

\paragraph{\cswap : Reversing product order}

Since $$\bfR\Exp(\bth)\bfR\tr=\bfR\exp(\hatx{\bth})\bfR\tr=\exp(\bfR\hatx{\bth}\bfR\tr)=\exp(\hatx{\bfR\bth})=\Exp(\bfR\bth),$$ then
%
\begin{align}
\Exp(\bth)\bfR &= \bfR\Exp(\bfR\tr\bth) \\
\bfR\tr\Exp(\bth)\bfR &= \Exp(\bfR\tr\bth) \\
\Exp(\bth)\Exp(\bphi) &= \Exp(\bphi)\Exp(\Exp(\bphi)\tr\bth) 
\end{align}


\paragraph{\cexpand, \csubst, \ccancel : Expand, substitute, cancel :} This happens when we expand or substitute a previously defined term, or when we cancel terms.

\paragraph{\tcom{$\oplus$}, \tcom{$\ominus$}, \tcom{(1)} : Apply definition :} This happens when we apply a particular definition or equation number.

\subsection{Common mistakes: Don'ts}

\begin{align}
\Exp(\bth_1+\bth_2) &\ne \Exp(\bth_1)\Exp(\bth_2) \\
\Log(\bfR_1\bfR_2) &\ne \Log(\bfR_1) + \Log(\bfR_2) \\
\bfJ_r(\bth) &\ne \bfI - \frac12\hatx{\bth} \\
\bfJ_r\inv(\bth) &\ne \bfI + \frac12\hatx{\bth} 
\end{align}
%
however, these hold approximately true for small angle vectors. See \csmall above.


\subsection{Examples}

\subsubsection{Example 1: $\bbR^3\times\bbR^3\to\bbR^3$} 

The rotation $f(\bth,\bfv) = \bfR(\bth)\,\bfv = \Exp(\bth)\,\bfv \in \bbR^3$ produces vectors of $\bbR^3$ from vectors $\bth,\bfv$ in $\bbR^3$. Its Jacobians are defined by \eqRef{equ:derivative_vector} and developed as
%
\begin{align*}
\dpar{\bfR(\bth)\bfv}{\bth} 
&= \lim_{\delta\bth\to0}\frac{\Exp(\bth+\delta\bth)\bfv-\Exp(\bth)\bfv}{\delta\bth} \\
\cJr
&= \lim_{\delta\bth\to0}\frac{\Exp(\bth)\Exp(\bfJ_r(\bth)\delta\bth)\bfv-\Exp(\bth)\bfv}{\delta\bth} \\
\csmall
&= \lim_{\delta\bth\to0}\frac{\Exp(\bth)(\bfI+\hatx{\bfJ_r(\bth)\delta\bth})\bfv-\Exp(\bth)\bfv}{\delta\bth} \\
\ccancel
&= \lim_{\delta\bth\to0}\frac{\Exp(\bth)\hatx{\bfJ_r(\bth)\delta\bth}\bfv}{\delta\bth} \\
\ccross
&= \lim_{\delta\bth\to0}\frac{-\Exp(\bth)\hatx{\bfv}\bfJ_r(\bth)\delta\bth}{\delta\bth} \\
&= -\bfR(\bth)\hatx{\bfv}\bfJ_r(\bth) 
\end{align*}
%
and
%
\begin{align*}
\dpar{\bfR(\bth)\bfv}{\bfv} 
&= \lim_{\partial\bfv\to0}\frac{\bfR(\bth)(\bfv+\partial\bfv)-\bfR(\bth)\bfv}{\partial\bfv} \\
\com{cancel} 
&= \bfR(\bth)
\end{align*}

\subsubsection{Example 2: $SO(3)\times\bbR^3\to\bbR^3$} 

The rotation $f(\bfR,\bfv) = \bfR\,\bfv \in \bbR^3$ produces vectors of $\bbR^3$ from elements $\bfR\in SO(3)$ and vectors in $\bbR^3$. The first Jacobian is defined by \eqRef{equ:jacobian_SO3_Rn} and developed as
%
\begin{align*}
\dpar{\bfR\bfv}{\bth} 
&\te \lim_{\delta\bth\to0}\frac{(\bfR\oplus\delta\bth)\bfv-\bfR\bfv}{\delta\bth} \\
\com{$\oplus$}
&= \lim_{\delta\bth\to0}\frac{\bfR\Exp(\delta\bth)\bfv-\bfR\bfv}{\delta\bth} \\
\csmall
&= \lim_{\delta\bth\to0}\frac{\bfR\tdot(\bfI+\hatx{\delta\bth})\bfv-\bfR\bfv}{\delta\bth} \\
\ccancel
&= \lim_{\delta\bth\to0}\frac{\bfR\hatx{\delta\bth}\bfv}{\delta\bth} \\
\ccross
&= \lim_{\delta\bth\to0}\frac{-\bfR\hatx{\bfv}\delta\bth}{\delta\bth} \\
&= -\bfR\hatx{\bfv} 
\end{align*}
%
The second Jacobian is defined by \eqRef{equ:derivative_vector} and trivially develops as,
%
\begin{align*}
\dpar{\bfR\bfv}{\bfv} 
&\te \lim_{\partial\bfv\to0}\frac{\bfR\tdot(\bfv+\partial\bfv)-\bfR\bfv}{\partial\bfv} \\
&= \bfR
\end{align*}

\subsubsection{Example 3: $SO(3)\times\bbR^3\to SO(3)$} 

The function $f(\bfR,\bw) = \bfR\Exp(\bw\dt)\in SO(3)$ produces elements of $SO(3)$ from elements in $SO(3)$ and vectors $\bw$ in $\bbR^3$. Its Jacobians are 
%
\begin{align*}
\dpar{\bfR\Exp(\bw\dt)}{\bth} 
&= \lim_{\delta\bth\to0}\frac{(\bfR\oplus\delta\bth)\Exp(\bw\dt)\ominus(\bfR\Exp(\bw\dt))}{\delta\bth} \\
\com{$\oplus,\ominus$}
&= \lim_{\delta\bth\to0}\frac{\Log\big((\bfR\Exp(\bw\dt))\inv \bfR\Exp(\delta\bth)\Exp(\bw\dt)\big)}{\delta\bth} \\
&= \lim_{\delta\bth\to0}\frac{\Log\big(\Exp(\bw\dt)\tr\bfR\tr \bfR\Exp(\delta\bth)\Exp(\bw\dt)\big)}{\delta\bth} \\
\ccancel
&= \lim_{\delta\bth\to0}\frac{\Log\big(\Exp(\bw\dt)\tr\Exp(\delta\bth)\Exp(\bw\dt)\big)}{\delta\bth} \\
\cswap
&= \lim_{\delta\bth\to0}\frac{\Log\big(\Exp(\Exp(\bw\dt)\tr\delta\bth)\big)}{\delta\bth} \\
\ccancel
&= \lim_{\delta\bth\to0}\frac{\Exp(\bw\dt)\tr\delta\bth}{\delta\bth} \\
&= \Exp(-\bw\dt) 
\end{align*}
%
and
%
\begin{align*}
\dpar{\bfR\Exp(\bw\dt)}{\bw} 
&= \lim_{\delta\bw\to0}\frac{\bfR\Exp((\bw+\delta\bw)\dt) \ominus (\bfR\Exp(\bw\dt)) }{\delta\bw} \\
\com{$\ominus$}
&= \lim_{\delta\bw\to0}\frac{\Log\big((\bfR\Exp(\bw\dt))\inv \, \bfR\Exp(\bw\dt+\delta\bw\dt)\big)}{\delta\bw} \\
\cJr
&= \lim_{\delta\bw\to0}\frac{\Log\big((\bfR\Exp(\bw\dt))\inv \bfR\Exp(\bw\dt)\Exp(\bfJ_r(\bw\dt)\delta\bw\dt)\big)}{\delta\bw} \\
\ccancel
&= \lim_{\delta\bw\to0}\frac{\Log\big(\Exp(\bfJ_r(\bw\dt)\delta\bw\dt)\big)}{\delta\bw} \\
\com{cancel}
&= \bfJ_r(\bw\dt)\dt
\end{align*}
%
%This can be seen in a easier way using the chain rule on $\bw\dt$,
%%
%\begin{align*}
%\dpar{\bfR\Exp(\bw\dt)}{\bw} 
%&= \dpar{\bfR\Exp(\bw\dt)}{\bw\dt}\dpar{\bf\dt}{\bw} \\
%&= \dpar{\bfR\Exp(\bw\dt)}{\bw\dt}\dt \\
%\cJr
%&= \bfJ_r(\bw\dt)\dt
%\end{align*}
%
Refer to the text for other developments, \eg~\secRef{sec:jac_first_delta}.



\subsubsection{Example 4: $SO(3)\times SO(3)\to SO(3)$}

The function $f(\bfR,\bfS) = \bfR\{\bftheta\}\,\bfS\{\bfphi\} \in SO(3)$ produces the concatenation of rotations, or rotation composition. Its Jacobians are
%
\begin{align*}
\dpar{\bfR\{\bftheta\}\,\bfS\{\bfphi\}}{\bftheta} 
&= \lim_{\delta\bftheta\to0}\frac{\Log\big((\bfR\bfS)\inv((\bfR\oplus\delta\bftheta)\bfS)\big)}{\delta\bftheta} \\
&= \lim_{\delta\bftheta\to0}\frac{\Log\big((\bfR\bfS)\tr(\bfR\Exp(\delta\bftheta)\bfS)\big)}{\delta\bftheta} \\
&= \lim_{\delta\bftheta\to0}\frac{\Log\big(\bfS\tr\bfR\tr\bfR\Exp(\delta\bftheta)\bfS\big)}{\delta\bftheta} \\
&= \lim_{\delta\bftheta\to0}\frac{\Log\big(\bfS\tr\Exp(\delta\bftheta)\bfS\big)}{\delta\bftheta} \\
\cswap
&= \lim_{\delta\bftheta\to0}\frac{\Log\big(\Exp(\bfS\tr\delta\bftheta)\big)}{\delta\bftheta} \\
&= \lim_{\delta\bftheta\to0}\frac{\bfS\tr\delta\bftheta}{\delta\bftheta} \\
&= \bfS\tr
\end{align*}
%
and
%
\begin{align*}
\dpar{\bfR\{\bftheta\}\,\bfS\{\bfphi\}}{\bfphi} 
&= \lim_{\delta\bfphi\to0}\frac{\Log\big((\bfR\bfS)\tr(\bfR(\bfS\oplus\delta\bfphi))\big)}{\delta\bfphi} \\
&= \lim_{\delta\bfphi\to0}\frac{\Log\big((\bfR\bfS)\tr(\bfR\bfS\Exp(\delta\bfphi))\big)}{\delta\bfphi} \\
&= \lim_{\delta\bfphi\to0}\frac{\Log\big(\Exp(\delta\bfphi)\big)}{\delta\bfphi} \\
&= \lim_{\delta\bfphi\to0}\frac{\delta\bfphi}{\delta\bfphi} \\
&= \bfI
\end{align*}





\section{Uncertainties in $SO(3)$}

\subsection{Description of uncertainty}
\subsubsection{Vector spaces}
\begin{align*}
\widetilde\bfx&=\bfx+\delta\bfx
\end{align*}
\subsubsection{$SO(3)$}
\begin{align*}
\widetilde\bfR&=\bfR\oplus\dth\te\bfR\Exp(\delta\bth) \\
\widetilde\bfq&=\bfq\oplus\dth\te\bfq\ot\Exp(\delta\bth)
\end{align*}

\subsection{Uncertainty propagation}

%
The uncertainty of the composition of two uncertain elements of $SO(3)$ can be derived as follows (we use the matrix form for convenience),
%
\begin{align*}
\widetilde\bfR = \bfR\Exp(\dth) 
&= \widetilde{\bfR_1\bfR_2} \\
&= \widetilde\bfR_1\widetilde\bfR_2 \\
\cexpand
&= \bfR_1\Exp(\dth_1)\bfR_2\Exp(\dth_2) \\
\cswap
&= \bfR_1\bfR_2\Exp(\bfR_2\tr\dth_1)\Exp(\dth_2) \\
\cJr
&\approx \bfR_1\bfR_2\Exp(\bfR_2\tr\dth_1+\bfJ_r\inv(\bfR_2\tr\dth_1)\dth_2) \\
&= \bfR\Exp(\bfR_2\tr\dth_1+\bfJ_r\inv(\bfR_2\tr\dth_1)\dth_2) 
\end{align*}
%
and therefore,
%
\begin{align}
\dth = \bfR_2\tr\dth_1+\bfJ_r\inv(\bfR_2\tr\dth_1)\dth_2
\end{align}
%
We can derive equivalent Jacobian matrices for the propagation. Since each partial derivative assumes null perturbations in  other variables, we have,
%
\begin{align}
\DTH_{\dth_1} \te \dpar{\bth}{\bth_1} &= \frac{\dth(\dth_2=0)}{\dth_1} = \bfR_2\tr \\
\DTH_{\dth_2} \te \dpar{\bth}{\bth_2} &= \frac{\dth(\dth_1=0)}{\dth_2} = \bfI
\end{align}
%
yielding
%
\begin{align}
\dth \approx \bfR_2\tr\dth_1+\dth_2
\end{align}
%
which constitutes a regular frame transformation operation. The covariances are propagated from $\bfR_1$ and $\bfR_2$ to $\bfR$ as 
%
\begin{align*}
\Sigma = \bfR_2\tr\,\Sigma_1\,\bfR_2 + \Sigma_2
\end{align*}

