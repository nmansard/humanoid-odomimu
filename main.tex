%%%%%%%%%%%%%%%%%%%%%%%%%%%%%%%%%%%%%%%%%%%%%%%%%%%%%%%%%%%%%%%%%%%%%%%%%%%%%%%%
%2345678901234567890123456789012345678901234567890123456789012345678901234567890
%        1         2         3         4         5         6         7         8

\documentclass[letterpaper, 10 pt, conference]{ieeeconf}  % Comment this line out if you need a4paper

%\documentclass[a4paper, 10pt, conference]{ieeeconf}      % Use this line for a4 paper

\IEEEoverridecommandlockouts                              % This command is only needed if 
                                                          % you want to use the \thanks command

\overrideIEEEmargins                                      % Needed to meet printer requirements.

% See the \addtolength command later in the file to balance the column lengths
% on the last page of the document

% The following packages can be found on http:\\www.ctan.org
\usepackage{graphicx} % for pdf, bitmapped graphics files
\usepackage{epsfig} % for postscript graphics files
%\usepackage{mathptmx} % assumes new font selection scheme installed
%\usepackage{times} % assumes new font selection scheme installed
\usepackage{amsmath} % assumes amsmath package installed
\usepackage{amssymb}  % assumes amsmath package installed
\usepackage{algorithm}
\usepackage{algorithmic}

\usepackage{color}
\usepackage{bm} 
\usepackage{customCommands}

%\usepackage[pdflatex]{hyperref}

% More macros
\newcommand{\bw}{{\bfomega}}
\newcommand{\bth}{{\bftheta}}
\newcommand{\bphi}{{\bfphi}}
\newcommand{\nth}{\norm{\bth}}
\newcommand{\ab}{{\bfa_b}}
\newcommand{\wb}{{\bw_b}}
\newcommand{\D}{\Delta}
\newcommand{\Dzero}{{\D^0}}
\newcommand{\Dp}{{\D\bfp}}
\newcommand{\Dv}{{\D\bfv}}
\newcommand{\Dth}{{\D\bth}}
\newcommand{\Dq}{{\D\bfq}}
\newcommand{\DR}{{\D\bfR}}
\newcommand{\DP}{{\D\bfP}}
\newcommand{\DV}{{\D\bfV}}
\newcommand{\DTH}{{\D\bfTheta}}
\newcommand{\Dw}{{\D\bw}}
\newcommand{\DW}{{\D\bfOmega}}
\newcommand{\dpp}{{\delta\bfp}}
\newcommand{\dv}{{\delta\bfv}}
\newcommand{\dth}{{\delta\bth}}
\newcommand{\dq}{{\delta\bfq}}
\newcommand{\dR}{{\delta\bfR}}
\newcommand{\dP}{{\delta\bfP}}
\newcommand{\dV}{{\delta\bfV}}
\newcommand{\dTH}{{\delta\bfTheta}}
\newcommand{\dw}{{\delta\bw}}

\newcommand{\te}{\triangleq}
\newcommand{\od}{\odot}


%%%%%%%%%%%%%%%%%%%%%%%%%%%%%%%%%%%%%%%%%%%
\title{\LARGE \bf
Odometry based on auto-calibrating inertial measurement units attached to the feet
}

\author{Dinesh Atchuthan$^{1}$, Angel Santamaria-Navarro$^{2}$, Nicolas Mansard$^1$, Olivier Stasse$^1$, Joan Sol\`a$^{1,2}$% <-this % stops a space
\thanks{$^{1}$ CNRS - LAAS, Toulouse, France, \tt {\small first.last@laas.fr}}%
\thanks{$^{2}$ IRI, UPC, Barcelona, \tt{\footnotesize \{asantamaria,jsola\}@iri.upc.edu}}
}

\begin{document}

\maketitle
\thispagestyle{empty}
\pagestyle{empty}

%%%%%%%%%%%%%%%%%%%%%%%%%%%%%%%%%%%%%%%%%%%%%%%%%%%%%%%%%%%%%%%%%%%%%%%%%%%%%%%%
\begin{abstract}
While the feet of many of our legged robots are equipped with force sensors, we often only have a rough idea of their exact location in space, neither of the dynamical effects (flexibility, inertia) acting on them.
However, inaccuracy of the foot location, trajectory or friction often result in rapid and unavoidable falls.
In this paper, we are considering the problem of accurately estimating the placement of the foot, either during the contact phase (to assert friction and no sliding) or during the flying phase (to compensate for flexibility in the actuation chain).
We propose to collocate an inertial measurement unit (IMU) directly on the foot in order to reconstruct its position.
The main difficulty is then to recover the observability of the IMU location by integrating all previous sensor measurement along with additional constraints coming from the contact and the kinematic chain.
More precisely, a graphical representation approach is presented to integrate IMU measurement, kinematic and contact information on a humanoid robot. 
Contrary to many state estimators recently proposed in the context of legged locomotion, we propose to rely on an information graph representation to handle observability consistency and other constraints.
We rely on the recent developments in the SLAM community where various and heterogeneous measurements are fused in a coherent mathematical framework.
The technical difficulty is to handle the size of the graph such that it is tractable in a limited window, in particular in the view of the high frequency of the IMU.
Forster et al. proposed to pre-integrate the most frequent data which are given by the IMU.
Based on this early work, we propose a detailed derivation of this pre-integration using quaternion instead of rotation matrices leading to faster computation and smaller memory foot-print.
We validate the concept by estimating the trajectory of the foot of the humanoid robot HRP-2 during various gaits, and show that we are able to accurately reconstruct some subtle effects, such as sliding effects during the contact and flexibility of the kinematic chain.

%% In this paper, a graphical representation approach is presented to integrate 
%% IMU measurement and kinematic information on a humanoid robot. Although state estimators using both
%% informations \cite{Johnson:jof:2016,Fallon:ichr:2014} have been already proposed, this method has several additional interests.
%% First it is allowing to estimate bias and initial position of the IMU thanks to the graph
%% approach. Second the concept has been extensively used in the SLAM community and integrating 
%% other measurements such as laser, range image and so on can be done in a coherent mathematical framework.
%% The technical difficulty is to handle the size of the graph such that it is tractable in a limited window.
%% \cite{forster2015imu} proposed to pre-integrate the most frequent data which are given by the IMU.
%% In this paper we propose a detailed derivation of this pre-integration using quaternion instead
%% of rotation matrices leading to faster computation and smaller memory foot-print.
%% The algorithm is tested on the HRP-2 humanoid robot to estimate
%% its flying foot trajectory and validated against motion capture data.
\end{abstract}


%%%%%%%%%%%%%%%%%%%%%%%%%%%%%%%%%%%%%%%%%%%%%%%%%%%%%%%%%%%%%%%%%%%%%%%%%%%%%%%%
% !TEX root = main.tex

\section{Introduction}\label{sec:intro}

\textit{Context: }
Indoor person localization is an open challenge in various situtations: location-based life improving services, firefighters localization and navigation, patients tracking 
motion monitoring, medical observation, accident monitoring \cite{pourhomayoun2012spatial}, mobility and independance of partially-sighted or blind persons, etc.
As GPS are not available indoor, and relying on a network of fixed sensors (cameras, RFID) also raised many open questions, an appealing way to localize a body in space is to use odometry information measured by embedded interial measurement units (IMU).
In this context, it is mandatory the integration is done while taking into account the biais of the IMU.
As the bias varies with time and conditions, it must be estimated on-line while integrating the measurements.
Furthermore, it is desirable that additional informations coming from other sensors can be integrating in the same estimation process.
Simliarly to the strategies adopted in simulatenous localization and mapping (SLAM), sparse measurements or additional information (e.g. coming from intermittent absolute localization, or from a sparse sensor network) would benefit to the localization process when available.
With these requirements coming from the context in mind, we propose to define an estimator based on graphical models, able to accurately and efficiently integrate inertial measurements while estimated the biases of the IMU. Thanks to the graphical model, the estimator will then be easily extended to fusion additional measurements coming from other sensors or additional prior information coming from the application.
This estimator is used to integrate the inertial measurements of a IMU attached to the foot of a pedestrian walking on structured terrain (flat floor or stairs).

\TODO{Add cover figure}
% \begin{figure}
% \centering
% \includegraphics[width=\linewidth]{./figures/cover-figure.pdf}
% 	\caption{D(from \cite{Carpentier:ICRA:2016}).
%  }
% 	\label{fig:cover}
% \end{figure}

\textit{Methodology: }
Graphical methods have been extensively used to implement such fusion strategies \cite{Thrun:ijrr:2006,Kaess:itro:2008}.
They are well-suited to gather information from sensors
and draw conclusions. The underlying principle is to consider that desptite all the information gathered from the sensors, we still have uncertainty about the true state of the world due to imperfections of the sensors.
Several states of the world can thus be considered as probable.
Relying on probabilictic formulations is a way find out the most probable one. Furthermore, graphical representation are able to accurately model 
complex estimation problems~\cite{koller2009probabilistic} in a versatile way. 

Graphical models have been used for large modeling estimation problems by means of sparse networks of constraints and particularly in
robotics where SLAM and visual odometry problems have reached a high degree of maturity in great part thanks to these tools.
The graphical representation also allows for the design of powerful nonlinear estimation solvers, which can be built taking into account the needs for accuracy, 
robustness and CPU-performance.

In order to keep the problem tractable and maintain real-time performance a key point is to avoid the graph to be too large for a given time window.
IMUs are challenging in this regards, as their high frequency measurements create large sets of data. 
Pre-integration of IMU measurements helps to reduce the size of the graph by summarizing 100 to 1000 measures into a single pre-integrated Bayesian node~\cite{LUPTON-09}.
Direct pre-integration leads to a dependency of the resulting node on the initial integration condition, implying to integrate again (and again) when an optimizer process the graph.
It was later suggested to make this pre-integrated data independant from the initial state where it was computed~\cite{forster2015imu}.
Consequently, the IMU measurements can simply be disgarded even when the initial ``pre-integration'' condition changes while the numerical solver optimized the maximum-likelihood trajectory.

\textit{Contibutions: }
In this paper, we follow a similar methodology. We define a graphical model where pre-integrated data are added at key-frame instants (about 2Hz), summarizing IMU measurements captured at about 1kHz.
The state that we want to estimate is the position, orientation and linear velocity (dimension 9) of the IMU attached to the foot, along with the accelerometer and gyroscope biaises (both varying in time -- dimension 6).
We also implement the prior knowledge that the foot lands horizontally, by adding Bayesian nodes when the foot lands and takes off.
This prior knowlegde make the considered state observable. 
We then use a numerical solver to maximize the likelihood of the measurements and prior knowledge along the past trajectory in a slinding window. 

A first contribution of this paper is to reformulate the method proposed in \cite{forster2015imu} from rotation matrix to quaternion representation
and give a detailed and simpler algebraic derivation.
The second contribution is to apply this method to the estimation of foot-pose during human walking.
We describe the implementation of this pre-integration scheme in a software implementation of the graphical method. This implementation would allow to simply add different sensor or additioanl priors for fusion strategies.
We compare the level of accuracy obtained with this method against other methods classicaly used in pedestrian localization.

We first describe the implementation of this pre-integration scheme in a software implementation of the graphical method designed to
allow users to easily gather information from different sensor for fusion strategies.
The gathered information are then used to get a graphical model-based formulation of the problem. 
Once the formulation is completed, a non-linear least squares optimizer is used to solve the problem and find the most probable solution in the least-square sense \cite{ceres-solver}.






% !TEX root = Chapter2.tex

\section{Graph-based inertial-kinematic odometry}

%We describe the inertial-kinematic odometry for legged robots, based on a graph model. 


%%%%%%%%%%%%%%%%%%%%%%%%%%%%%%%%%%%%%%%%%%%%%%%%%%%%%%%%%%%%

\subsection{Quaternion rotations}

We define the quaternion-by-vector product $\od$ so that
%
\begin{align}
\bfq\od\bfv \te \bfq\ot\bfv\ot\bfq^*
~,
\end{align}
%
where $\bfq^*$ denotes the dual quaternion of $\bfq$. 
The symbol $\ot$ stands for the product of quaternions and $\od$ corresponds to the quaternion-by-vector which performs a 3D rotation of an input vector $\bfv$. 
Notice that if $\bfR$ is the rotation matrix equivalent to the quaternion $\bfq$, then 
%
$\bfq\od\bfv = \bfR\,\bfv$. 
%
This straightforward equivalence
enables us to define all the forthcoming IMU pre-integration algebra in a way that allows a direct transcription between the $S^3$  and $SO(3)$ spaces of representation.

%%%%%%%%%%%%%%%%%%%%%%%%%%%%%%%%%%%%%%%%%%%%%%%%%%%%%%%%
\subsection{Graph-based iterative optimization}

In graph-based optimization, the problem is represented as a graph, where the nodes refer to the variables, and the edges, called factors, represent the geometrical constraints that link between variables produced by the measurements.
%
The state $\bfx$ is modeled as a multi-variate Gaussian distribution, and in our case it includes foot poses and velocities $(\bfp,\bfq,\bfv)$ and IMU biases $(\bfa_b,\bw_b)$ at selected keyframes along the trajectory (see \figRef{fig:factor_graph}).
%
For each factor, we can define an error or a residual $\bfr$ as the discrepancy between a measurement $\bfz$ and its expectation given the involved state variables,
%
\begin{equation}
    \bfr(\bfx) = h(\bfx) + \bfv - \bfz, \qquad \bfv \sim \mathcal \cN (0, \bfOmega\inv)\label{eq:error}
\end{equation}
%
being $h(\bfx)$ the sensor measurement model and $\bfOmega$ the information matrix of the measurement Gaussian noise $\bfv$.
Importantly, the functions $h(\bfx)$ and $\bfr(\bfx)$ are very sparse, since only a small handful of blocks of $\bfx$ are involved in each factor, which results in a loosely connected graph.
In case of variables defined on manifolds, such as quaternions or rotation matrices, we must rewrite~\eqref{eq:error} as \mbox{$\bfr(\bfx) = (h(\bfx) \oplus \bfv) \ominus \bfz$}.
%, with $\bfJ_k = \partial (h_k(\bfx)\ominus\bfz_k)/\partial \Delta \bfx$. 
The $\oplus$ and $\ominus$ symbols correspond to the addition and subtraction operators on the manifold
%, as required for certain state blocks intervening in the factor 
(see \eg~Eq.~\eqRef{equ:imu_residual} in \secRef{sec:imu}, or Eq.~\eqsRef{equ:plus_q}{equ:minus_R} in \appRef{sec:derivatives_SO3}).

The maximum a posteriori estimation is obtained by iteratively minimizing the Mahalanobis squared norm of all linearized errors
%
\begin{align}
  \Delta \bfx^* &= \argmin_{\Delta \bfx} \sum_k \norm{ \bfr_k(\breve\bfx) + \bfJ_k \Delta \bfx }_{\bfOmega_k^{-1}}^2 \label{eq:LeastSquares}
\end{align}
%
being $\breve\bfx$ the state estimate at the current iteration, and $\bfJ_k$ the Jacobian of the $k$-th residual $\bfr_k(\bfx)$ (with \mbox{$\bfJ_k = \partial (h_k(\bfx)\ominus\bfz_k)/\partial \Delta \bfx$} in the case of variables lying on a manifold) and $\bfOmega_k$ is the information matrix of the $k$-th measurement.
%
Current methods use the Cholesky~\cite{Kummerle_icra11,ila_ijrr17} or the QR~\cite{Dellaert_ijrr06,Kaess_ijrr11} matrix factorizations to solve for $\Delta\bfx^*$, which is then used to update the current state estimate. 
The process is iterated until convergence.
%
Incremental methods~\cite{ila_ijrr17,Kaess_ijrr11} update the problem directly on the factorized matrices, obtaining important speed-ups.


%%%%%%%%%%%%%%%%%%%%%%%%%%%%%%%%%%%%%%%%%%%%%%%%%%%%%%%%%

%Estimation is realized with a tool in ongoing development \textbf{(cite WOLF ?)} and using Google Ceres non-linear least squares optimizer. This optimization is run on data given by the front-end part of the framework,
%which aim is to formalize the problem according to the graph-based model for aforementioned reasons. This representation, illustrated in \figRef{fig:factor_graph}, 
%gives a human-readable way to easily visualize optimized states (circles) linked to constraining information during the optimization process (factors drawn as squares). Constraints are set by the front-end process using measurements from sensors.
%Thus, it is possible to constrain states with several factors provided by different sensors leading to a fusion. Factor play a key role since they are used on top of previous states to predict next ones and compute the error with the current states.
%The optimal states are computed so that the overall residual is minimized in a least squares meaning.


\begin{figure}[tb]
\centering
\includegraphics[scale=0.8]{figures/graph_exploded.pdf}
\caption{
%{\bf Top}: 
Detailed factor graph for the initial keyframe and two steps. 
\emph{Circles}: state blocks for position ($\bfp$), orientation quaternion ($\bfq$), velocity ($\bfv$), accelerometer bias ($\bfa_b$), gyrometer bias ($\bw_b$). 
\emph{Orange}: initial pose factor. 
\emph{Red}: kinematic factor (deduced from additional sensors). 
\emph{Purple}: zero-velocity factor. 
\emph{Green}: IMU's delta pre-integration factor. 
\emph{Blue}: bias drift factor. 
\emph{Cyan}: bias absolute factor. 
%{\bf Bottom}: Equivalent factor graph exhibiting one aggregate state block $\bfx=(\bfp,\bfq,\bfv,\bfa_b,\bw_b)$ for each key-frame. All factors are affecting exactly the same variables as in the Top graph.
}
\label{fig:factor_graph}
\end{figure}



%%%%%%%%%%%%%%%%%%%%%%%%%%%%%%%%%%%%%%%%%%%%%%%%%%%%%
%\subsection{Graph-based inertial-kinetic odometry}

\subsection{Keyframe variables}

During a biped walk, we take profit of certain situations where precise and reliable assumptions can be made. For example, the  foot velocity is null during its support phase. At these selected instants, we create the keyframes that will produce a chain of states. These states are linked by the measurements, forming our factor graph (\figRef{fig:factor_graph}). Each keyframe $\bff_i$ contains the following state blocks: the foot's position, velocity and orientation data, plus the IMU's accelerometer and gyrometer biases,
%
\begin{align}
\bff_i = \begin{bmatrix}
\bfp_i & \bfv_i & \bfq_i & \bfa_{b,i} & \bw_{b,i}
\end{bmatrix}\tr
~.
\end{align}
%


%%%%%%%%%%%%%%%%%%%%%%%%%%%%%%%%%%%%%%%%%%%%%%%%%%%%%
\subsection{Description of factors}

The types of factor considered in our graph are illustrated in \figRef{fig:factor_graph}. 
Each factor $k$ requires its own information matrix~$\bfOmega_k$, and its residual function $\bfr_k(\bfx)$. These residual functions are detailed hereafter.

\subsubsection{Absolute factors}

These include initial position and orientation (orange in the figure), zero velocity (purple), and bias magnitude (cyan). Each residual depends on a single state block, which is compared against a reference $\bfz_k$,
%
\begin{align}
\bfr_k(\bfphi_i) = \bfphi_i - \bfz_k
\end{align}
%
where $\bfphi_i$ is one among $\{\bfp_i,\bfv_i,\bfa_{b,i},\bw_{b,i}\}$. 
For the quaternion we implement the residual using the operator $\ominus$ on the sphere of dimension $3$ manifold, denoted $S^3$ (see \eqRef{equ:minus_q} in \appRef{sec:derivatives_SO3} for further details),
%
\begin{align}
\bfr_k(\bfq_i) = \bfq_i\ominus\bfz_k = \Log(\bfz_k^*\ot\bfq_i)
\end{align}
%

\subsubsection{Bias drift factors (blue)}

These are relative factors that allow the bias estimates to drift with time at a controlled rate. Each bias drift residual depends on two state blocks, namely
%
\begin{align}
\begin{split}
\bfr(\bfa_{b,i}, \bfa_{b,j}) &= \bfa_{b,j} - \bfa_{b,i} 
\\
\bfr(\bw_{b,i} , \bw_{b,j})  &= \bw_{b,j}  - \bw_{b,i}
\end{split}
\end{align}

\subsubsection{Complementary factors (red)}

These relate position and orientation between two consecutive steps as it can be provided by other sensors than IMU or methods using human walking specificities,
%
\begin{align}
\bfr(\bfp_i,\bfq_i,\bfp_j,\bfq_j) = \begin{bmatrix}
\bfq_i^*\od(\bfp_j-\bfp_i) - \bfy_k \\
\Log(\bfz_k^*\ot\bfq_i^*\ot\bfq_j)
\end{bmatrix}
\end{align}
%
where $\bfy_k$ and $\bfz_k$ are respectively the relative position and quaternion measurements.

\subsubsection{IMU pre-integrated factors (green)}

These factors are by far the most complex ones and are described in details in the next section.

% !TEX root = main.tex


% More macros
\newcommand{\bw}{{\bfomega}}
\newcommand{\bth}{{\bftheta}}
\newcommand{\bphi}{{\bfphi}}
\newcommand{\nth}{\norm{\bth}}
\newcommand{\ab}{{\bfa_b}}
\newcommand{\wb}{{\bw_b}}
\newcommand{\D}{\Delta}
\newcommand{\Dzero}{{\D^0}}
\newcommand{\Dp}{{\D\bfp}}
\newcommand{\Dv}{{\D\bfv}}
\newcommand{\Dth}{{\D\bth}}
\newcommand{\Dq}{{\D\bfq}}
\newcommand{\DR}{{\D\bfR}}
\newcommand{\DP}{{\D\bfP}}
\newcommand{\DV}{{\D\bfV}}
\newcommand{\DTH}{{\D\bfTheta}}
\newcommand{\Dw}{{\D\bw}}
\newcommand{\DW}{{\D\bfOmega}}
\newcommand{\dpp}{{\delta\bfp}}
\newcommand{\dv}{{\delta\bfv}}
\newcommand{\dth}{{\delta\bth}}
\newcommand{\dq}{{\delta\bfq}}
\newcommand{\dR}{{\delta\bfR}}
\newcommand{\dP}{{\delta\bfP}}
\newcommand{\dV}{{\delta\bfV}}
\newcommand{\dTH}{{\delta\bfTheta}}
\newcommand{\dw}{{\delta\bw}}

\newcommand{\te}{\triangleq}
\newcommand{\od}{\odot}

% Helper /keys/ 
\newcommand{\tcom}[1]{{\footnotesize/\texttt{#1}/} }
\newcommand{\com}[1]{{\footnotesize/\texttt{#1}/~} }
\newcommand{\cdef}{\com{def}}
\newcommand{\cchain}{\com{chain}}
\newcommand{\ccross}{\com{cross}}
\newcommand{\cJr}{\com{Jr}}
\newcommand{\csmall}{\com{small}}
\newcommand{\cswap}{\com{swap}}
\newcommand{\ctrans}{\com{trans}}
\newcommand{\clog}{\com{Log}}
\newcommand{\clim}{\com{lim}}
\newcommand{\ccancel}{\com{cancel}}
\newcommand{\cexpand}{\com{expand}}
\newcommand{\csubst}{\com{subst}}

\section{IMU pre-integration in S3 and SO(3)}


%%%%%%%%%%%%%%%%%%%%%%%%%%%%%%%%%%%%%%%%%%%%%%%%%%%%%%%%%%%%
\subsection{Quaternion rotations}

We define the quaternion-by-vector product $\od$ so that
%
\begin{align}
\bfq\od\bfv \te \bfq\ot\bfv\ot\bfq^*
~,
\end{align}
%
where the symbol $\ot$ indicates the quaternion product.
That is, quaternion-by-vector products using the symbol $\od$ perform 3D rotations. 
Notice that if $\bfR$ is the rotation matrix equivalent to the quaternion $\bfq$, then $\bfq\od\bfv = \bfR\,\bfv$ and $\bfq^*\od\bfv = \bfR\tr\,\bfv$. 
In practice, the operator $\od$ may be implemented as ~$\bfq\od\bfv = \bfq\ot\bfv\ot\bfq^*$ ~or~ $\bfq\od\bfv = \bfR\,\bfv$, whichever is faster.


%%%%%%%%%%%%%%%%%%%%%%%%%%%%%%%%%%%%%%%%%%%%%%%%%%%%%%%%%%%%
\subsection{Exp and Log maps}

We use vectorized versions of the exponential and logarithmic maps in $S3$ (the group of unit quaternions) and $SO(3)$ (the group of rotation 3-matrices), and mark them with capitalized names $\Exp()$ and $\Log()$. They operate directly on the vector space $\bbR^3$, and use either quaternions as the representation of $S3$,
%
\begin{subequations}
\begin{align}
\bfq
&= \Exp(\bth) \te \begin{bmatrix}
\cos(\theta/2) \\ \bfu\sin(\theta/2)
\end{bmatrix}\\ 
\theta\bfu &= \Log(\bfq) \te 2\,\qv\frac{\arctan({\norm{\qv},q_w})}{\norm{\qv}}
~,
\end{align}
\end{subequations}
%
or rotation matrices as the representation of $SO(3)$, 
%
\begin{subequations}
\begin{align}
\bfR
&= \Exp(\bth) \te \bfI + \sin\theta\hatx{\bfu} + (1-\cos\theta)\hatx{\bfu}^2~ \label{equ:rodrigues} \\ 
\theta\bfu &= \Log(\bfR) \te \frac{\theta(\bfR-\bfR\tr)^\vee}{2\sin\theta} 
~,
\end{align}
\end{subequations}
%
with $\theta=\cos\inv\left(\frac{\trace(\bfR)-1}{2}\right)$,
and where $\bullet^\vee$, known as the \emph{vee} operator, is the inverse of the \emph{skew} operator $\hatx{\bullet}$. 
Their exact form ($\bfq$ or $\bfR$) is always clear by the context.




%%%%%%%%%%%%%%%%%%%%%%%%%%%%%%%%%%%%%%%%%%%%%%%%%%%%%%%%%%%%

\subsection{State integration}

We define the world-referenced states of position, velocity, and orientation quaternion, $\bfx\!=\!(\bfp,\bfv,\bfq)$. 
Their time evolution is governed by the kinematic equation,
%
\begin{align}\label{equ:cont_basic}
\dot\bfp &= \bfv \\
\dot\bfv &= \bfq\od\bfa + \bfg \\
\dot\bfq &= \frac12\bfq\ot\bw 
\end{align}
%
where we identify $\bfb=(\bfa,\bw)$ as the \emph{body magnitudes}, that is, the magnitudes of acceleration and angular rate in the IMU reference frame.
Assuming constant body magnitudes within the interval $\dt\te t_k-t_j$, we have the discrete-time relation
%
\begin{align}\label{equ:g_second_order_integration}
\begin{split}
\bfp_{k} &= \bfp_j + \bfv_j\dt  + \frac12\bfg\dt^2 + \frac12\bfq_j\od\bfa_j\dt^2 \\
\bfv_{k} &= \bfv_j + \bfg\dt + \bfq_j\odot\bfa_j\dt \\
\bfq_{k} &= \bfq_j\ot\Exp(\bw_j\dt/2) 
\end{split}
\end{align}


%%%%%%%%%%%%%%%%%%%%%%%%%%%%%%%%%%%%%%%%%%%%%%%%%%%%%%%%%%%%
\subsection{Delta definitions}

Consider a non-rotating reference frame that is free-falling at the acceleration of gravity. An ideal (unbiased and noiseless) IMU glued to this frame would measure null accelerations and rotations. Any non-null measurements would imply motion of the IMU \wrt this frame.

The deltas $\D\!=\!(\Dp,\Dv,\Dq)$ are defined as the motion (position, velocity, orientation) of a body \wrt a non-rotating frame that is free-falling at the acceleration of gravity $\bfg$. At $t=t_i$, this frame was at position $\bfp_i$ and orientation $\bfq_i$, and moving at velocity $\bfv_i$. Then, the deltas $\D_{ij}=\D(\bfx_i,\bfx_j)$ at time $t=t_j$ \wrt time $t=t_j$ respond to the expression,
%
\begin{align}\label{equ:delta_definition}
\begin{split}
\Dp_{ij} &= \bfq_i^*\od\Big(\bfp_j - \bfp_i - \bfv_i\Dt_{ij} - \frac12\bfg\Dt_{ij}^2\Big) \\
\Dv_{ij} &= \bfq_i^*\od(\bfv_j - \bfv_i - \bfg\Dt_{ij}) \\
\Dq_{ij} &= \bfq_i^*\ot\bfq_j 
\end{split}
\end{align}
%
where $\Dt_{ij} \te t_j - t_i$. Interestingly, the deltas form a group under the composition operator $\D_{ik}\te\D_{ij}\oplus\D_{jk}$, defined by,
%
\begin{align} \label{equ:g_composition}
\begin{split}
\Dp_{ik} 
&= \Dp_{ij} + \Dv_{ij}\Dt_{jk} + \Dq_{ij}\od\Dp_{jk} \\
\Dv_{ik} 
&= \Dv_{ij} + \Dq_{ij}\od\Dv_{jk} \\
\Dq_{ik} 
&= \Dq_{ij}\ot\Dq_{jk} 
\end{split}
\end{align}
%
with identity $\D_0=[(0,0,0),(0,0,0),(1,0,0,0)]$, and inverse $\D_{ji}\te\D_{ij}\inv$ shuch that $\D\inv\op\D=\D\oplus\D\inv=\D_0$ (whose expression not given for space reasons).%, given by,
%
%\begin{align}
%\begin{split}
%\Dp_{ji} 
%&= -\Dq_{ij}^*\od(\Dp_{ij} - \Dv_{ij}\Dt_{ij}) \\
%\Dv_{ji} 
%&= -\Dq_{ij}^*\od\Dv_{ij} \\
%\Dq_{ji} 
%&= \Dq_{ij}^*
%~.
%\end{split}
%\end{align}



%%%%%%%%%%%%%%%%%%%%%%%%%%%%%%%%%%%%%%%%%%%%%%%%%%%%%%%%%%%%
\subsection{Incremental delta pre-integration}

Substituting the integration equation \eqRef{equ:g_second_order_integration} in the delta definitions \eqRef{equ:delta_definition}, we obtain the incremental delta pre-integration,
%
\begin{align}\label{equ:g_second_order_pre-integration}
\begin{split}
\Dp_{ik} 
&= \Dp_{ij} + \Dv_{ij}\dt + \frac12\Dq_{ij}\od\bfa_j\dt^2 \\
\Dv_{ik} 
&= \Dv_{ij} + \Dq_{ij}\od\bfa_j\dt \\
\Dq_{ik} 
&= \Dq_{ij}\ot\Exp(\bw_j\dt) 
\end{split}
\end{align}
%
which, interestingly, expresses the motion of a body \emph{in an inertial frame} under constant acceleration and rotation rate.
Noticing that we can define a proper delta $\delta_{jk}$ from the body magnitudes $\bfb_j=(\bfa_j,\bw_j)$ at time $t_j$,
%
\begin{align}\label{equ:delta_creation}
\begin{split}
\dpp_{jk} &= \frac12\bfa_j\dt^2 \\
\dv_{jk} &= \bfa_j\dt \\
\dq_{jk} &= \Exp(\bw_j\dt)
\end{split}
\end{align}
%
and therefore the integration \eqRef{equ:g_second_order_pre-integration} can be expressed as, 
%
\begin{align}\label{equ:composition_compact}
\D_{ik}=\D_{ij}\oplus\delta_{jk}
~,
\end{align}
%
using  the composition \eqRef{equ:g_composition}. In the following, we will identify $\D$ with the pre-integrated delta and $\delta$ with the current delta.


%%%%%%%%%%%%%%%%%%%%%%%%%%%%%%%%%%%%%%%%%%%%%%%%%%%%%%%%%%%%
\subsection{Jacobians}

\newcommand{\jac}[2]{{\bfJ^{#1}_{#2}}}

Notation: 
We note all Jacobians with $\jac{y}{x}\te\dpar{y}{x}$. 



\subsubsection{Jacobians of the body magnitudes}

We consider biased and noisy measurements of the body magnitudes, so that,
%
\begin{align}
\begin{split}
\bfa &= \bfa_m - \ab + \bfa_n \\
\bw &= \bw_m - \wb + \bw_n 
~,
\end{split}
\end{align}
%
with $\bullet_m$ the measurements, $\bullet_b$ the biases, and $\bullet_n$ the noises.
We have immediately
%
\begin{align}
\jac{\bfb}{\bfb_m}&=\bfI_6 & \jac{\bfb}{\bfb_b}&=-\bfI_6 & \jac{\bfb}{\bfb_n}&=\bfI_6
~.
\end{align}

\subsubsection{Jacobians of the current delta}
\label{sec:jac_data}

The involved operations are the delta creation \eqRef{equ:delta_creation} whose Jacobians are mostly obtained by simple inspection,
%
\begin{align}\label{equ:jacobian_current_delta}
\jac{\delta}{\bfb} =
\begin{bmatrix}
\frac12\bfI\dt^2 	& \bf0 \\
\bfI\dt 			& \bf0 \\
\bf0 				& \bfJ_r(\bw\dt)\dt
\end{bmatrix}
\end{align}
%
We develop the lower-right block $\bfJ^\dth_\bw$ as follows,
%
\begin{align*}
\widetilde\dq \te \dq\ot\Exp(\partial\dth) 
&= \Exp((\bw+\partial\bw)\dt) \\
\cJr &= \Exp(\bw\dt)\ot\Exp(\bfJ_r(\bw\dt)\partial\bw\dt) \\
&= \dq \ot \Exp(\bfJ_r(\bw\dt)\partial\bw\dt)
~,
\end{align*}
%
which leads to 
$\partial\dth = \bfJ_r(\bw\dt)\partial\bw\dt$ 
and thus \eqRef{equ:jacobian_current_delta}.



\subsubsection{Jacobians of the commposition $\D^+=\D\op\delta$}

We differentiate the delta composition \eqRef{equ:composition_compact} detailed in \eqRef{equ:g_composition},
%
\begin{align}
\jac{\D^+}{\D} = \begin{bmatrix}
\bfI  & \bfI\dt & - \DR_{ij}  \hatx{\dpp_{jk}}  \\
\bf0  & \bfI    & - \DR_{ij}  \hatx{\dv_{jk}} \\
\bf0  & \bf0    &   \dR_{jk}\tr 
\end{bmatrix}
\end{align}
%
\begin{align}
\jac{\D^+}{\delta} = \begin{bmatrix}
\DR_{ij} & \bf0     & \bf0 \\
\bf0     & \DR_{ij} & \bf0 \\
\bf0     & \bf0     & \bfI  
\end{bmatrix}
\end{align}



%%%%%%%%%%%%%%%%%%%%%%%%%%%%%%%%%%%%%%%%%%%%%%%%%%%%%%%%%%%%
\subsection{Delta covariance integration}

Let $\bfQ$ be the covariance of the pre-integrated delta, and $\bfN$ that of the measurement noise. We have simply,
%
\begin{align}
\bfQ^+ = \jac{\D^+}{\D}\bfQ\,\,\jac{\D^+}{\D}\tr + \jac{\D^+}{\bfb_n}\bfN\,\jac{\D^+}{\bfb_n}\tr
\end{align}
%
where $\jac{\D^+}{\bfb_n}$ is the noise Jacobian, obtained with the chain rule,
%
\begin{align}
\jac{\D^+}{\bfb_n} = \jac{\D^+}{\delta} \cdot \jac{\delta}{\bfb} \cdot \jac{\bfb}{\bfb_n} = \jac{\D^+}{\delta} \cdot \jac{\delta}{\bfb}
~.
\end{align}


%%%%%%%%%%%%%%%%%%%%%%%%%%%%%%%%%%%%%%%%%%%%%%%%%%%%%%%%%%%%
\subsection{Delta correction with new bias}

Let $\ol\D$ and $\ol{\bfb_b}$ be respectively the pre-integrated delta and the bias values used during pre-integration. Should the bias estimates be updated to new values $\bfb_b$, we need to update the delta accordingly. We do so with the linearized update,
%
\begin{align}\label{equ:delta_correction}
\D = \ol\D + \jac{\D}{\bfb_b}(\bfb_b - \ol{\bfb_b})
~,
\end{align}
%
where $\jac{\D}{\bfb_b}$ is the bias Jacobian, obtained recursively using also the chain rule,
%
\begin{align*}
\jac{\D_{ik}}{\bfb_b} 
&= 
\jac{\D_{ik}}{\D_{ij}} \cdot \jac{\D_{ij}}{\bfb} \cdot \jac{\bfb}{\bfb_b} 
+ 
\jac{\D_{ik}}{\delta_{jk}} \cdot \jac{\delta_{jk}}{\bfb} \cdot \jac{\bfb}{\bfb_b} \\
&= \jac{\D_{ik}}{\D_{ij}} \cdot \jac{\D_{ij}}{\bfb_b} - \jac{\D_{ik}}{\delta} \cdot \jac{\delta}{\bfb}
~,
\end{align*}
%
which we might note for clarity,
%
\begin{align}
\jac{\D}{\bfb_b}|_{ik} = \jac{\D^+}{\D}\jac{\D}{\bfb_b}|_{ij} - \jac{\D^+}{\delta}\jac{\delta}{\bfb}
\end{align}


%%%%%%%%%%%%%%%%%%%%%%%%%%%%%%%%%%%%%%%%%%%%%%%%%%%%%%%%%%%%
\subsection{State reconstruction}

At any time $j$ we can recover the state estimate $\bfx_j$ given the state estimate $\bfx_i$ and the (corrected) delta $\D_{ij}$,
%
\begin{align} \label{equ:reconstruction}
\begin{split}
\bfp_j &= \bfp_i + \bfv_i\Dt_{ij} + \frac12\bfg\Dt_{ij}^2 + \bfq_i\od\Dp_{ij} \\
\bfv_j &= \bfv_i + \bfg\Dt_{ij} + \bfq_i\od\Dv_{ij} \\
\bfq_j &= \bfq_i\ot\Dq_{ij}   
\end{split}
\end{align}



%%%%%%%%%%%%%%%%%%%%%%%%%%%%%%%%%%%%%%%%%%%%%%%%%%%%%%%%%%%%
\subsection{Residual}

Residual computation for the optimization in factor graphs requires (see \figRef{fig:factor_graph}) the state estimates $\bfx_i$ and $\bfx_j$, the pre-integrated delta $\ol{\D_{ij}}$, the bias used during pre-integration $\ol{\bfb_{b,i}}$, the pre-integrated bias Jacobian $\jac{\D}{\bfb_b}$, and the current bias estimates, $\bfb_{b,i}$. The process is best understood if split into smaller steps. We first compute a corrected delta $\D_{ij}$ using \eqRef{equ:delta_correction}; then compute a predicted delta $\widehat\D_{ij}$ using \eqRef{equ:delta_definition}; and finally compute the residual with
%
\begin{align}
\bfr_{ij} 
= \begin{bmatrix}
\Dp_{ij}-\widehat\Dp_{ij} \\
\Dv_{ij}-\widehat\Dv_{ij} \\
\Log(\widehat\Dq_{ij}^*\ot\Dq_{ij})
\end{bmatrix} 
\in \bbR^9
~.
\end{align}





% !TEX root = main.tex

\section{Experiments} \label{sec:experiments}

Trajectory estimation is naturally more effective with the use of a good IMU. IMU technology is advanced at such point that 
one can find those systems for a range of price going from low cost to thousands of dollars for space-navigation or military purposes.
The performances of an IMU are caracterized by several factors from which we can mainly cite biases random walk and noises standard deviation
as examples.

We chose to use a low-cost IMU in our application to realize the feasibility of our method. For this purpose we selected the 
MPU6050 from invensense combining both an accelerometer and a gyroscope and extensively used in the open source community.

Several experiments were designed to check that the trajectory estimation is feasible.

\subsection{Method}
\subsubsection{bias auto-calibration}
A naive way to estimate the trajectory is to simply integrate measurements on top of the initial state vector whose variables would all be optimized. 
This may be theoretically possible with a well characterized IMU, meaning that the bias components will be known and parts of the state vector.
However, since we want our experiment to allow a calibration of the sensor, this solution is not accepted and we need a setup allowing a correct calibration of the sensor.

The parameters we need to calibrate for a correct integration are the biases on top of which we integrate the incoming data of the IMU.
The time varying property of the bias is a critical point to consider in order to avoid large deviations. This calibration is made possible
by dependencies of the delta pre-integration ($\Delta P(ab), \Delta V(ab), \Delta Q(\omega b)$). We understand that the fusion of the IMU with another odometer giving both
orientation and position increments helps to determine the biases. However, the least squares formulation makes possible the transfer of some parts
of the velocities in the biases or the opposite operation during the residual minimization phase. One way to get rid of these is to use experimental conditions to constrain the states, in the case of legged
locomotion we use the property of having null velocity for the feet once they reach the ground. Adding this knowledge to the graph allows for a
better estimation of the biases and trajectory.

\subsubsection{Experimental procedures}

We need to define a series of experimental procedures that will allow us to check different parts of the results :

First of all, the final estimated position and orientation is checked using a 3D printed dock for the IMU containing two possible positions for the sensor.
This is used as a unit test in which the final odometer is imposed by the printed structure.

Then we need to make sure that the intermediate positions are correct, increasing this way our confidence on correct velocity estimations.
Such a test implies the need of a ground truth to check our estimations against it. For this purpose we use a Motion Capture (abbreviated as MOCAP) system publishing both
pose and orientation of the IMU in the system's frame at 120 Hz. The MOCAP is a tracking system looking for specific markers in space
and trying to reconstruct previously defined objects with these. Once the object is identified, we then get the position and orientation of the barycenter defined from the markers.
A first calibration process was used to make MOCAP's and IMU's axis match and to get the pose of the IMU instead of the barycenter of the object.

Finally, we will run the application on HRP2 robot.


\subsection{Results}
% !TEX root = main.tex

\section{Conclusion}

We have presented a method to measure foot movement during robot biped locomotion. We fuse information from an IMU attached to the foot, the kinematic chain measured by the robot encoders, and available knowledge extracted from the gait phases, such as zero velocity and IMU bias dynamics. We used nonlinear optimization techniques based on factor graphs, which has proved to be a flexible and powerful fusion framework. For this, we have revised the IMU pre-integration theory, and proposed an implementation in the quaternions manifold, with simpler derivations than previous works, and with physical interpretations, which we believe go in the direction of improving the clarity of the method.


Results showed that this estimation method is able to detect subtle movements that are not measurable using the encoders alone, such as a sliding foot condition at the beginning or end of the contact phase. These slides, if undetected, have a dramatic negative impact on the overall robot odometry.

% !TEX root = main.tex

%%%%%%%%%%%%%%%%%%%%%%%%%%%%%%%%%%%%%%%%%%%%%%%%%%%%%%%%%%%%
\appendix

\section{Definition of the derivatives }

\subsection{The additive and subtractive operators in $SO(3)$}

In vector spaces $\bbR^n$, the addition and subtraction operations are performed with the regular sum `$+$' and minus `$-$' operations.
In $SO(3)$ this is not possible, but equivalent operators are needed for establishing a proper calculus corpus. 

We thus define the plus and minus operators, $\oplus,\ominus$, between elements $\sR\in SO(3)$, and elements $\bth\in\bbR^3$ of the tangent space at $\sR$, as follows.

\paragraph{The plus operator.}
The `plus' operator $\oplus:SO(3)\times\bbR^3\to SO(3)$ produces an element $\sS$ of $SO(3)$ which is the result of composing a reference element $\sR$ of $SO(3)$ with a (often small) rotation specified by a vector of $\bth\in\bbR^3$ in the vector space tangent to the reference element $\sR$,
%
\begin{align}
\sS = \sR\oplus \bth &\te \sR\circ\Exp(\bth) && \sR,\sS\in SO(3),~ \bth\in\bbR^3 
\end{align}
%
Notice that this operator may be defined for any representation of $SO(3)$. In particular, for the quaternion and rotation matrix we have,
%
\begin{align}
\bfq_\sS &= \,\bfq_\sR\oplus\bth = \bfq_\sR\ot\Exp(\bth) \\
\bfR_\cS &= \bfR_\sR\oplus \bth = \bfR_\sR\Exp(\bth) 
\end{align}

\paragraph{The minus operator.}
The `minus' operator $\ominus:SO(3)\times SO(3)\to\bbR^3$ is the inverse of the above. It returns the vectorial angular difference $\bth\in\bbR^3$ between two elements of $SO(3)$. This difference is expressed in the  vector space tangent to the reference element $\sR$, 
%
\begin{align}
\bth=\sS\ominus \sR
&\te \Log(\sR\inv \circ \sS)     && \sR,\sS\in SO(3),~ \bth\in\bbR^3  
\end{align}
%
which for the quaternion and rotation matrix reads,
%
\begin{align}
\bth &= \,\,\bfq_\sS\ominus\bfq_\sR\, = \Log(\bfq_\sR^*\ot\bfq_\sS)                      \\
\bth &= \bfR_\sS\ominus\bfR_\sR = \Log(\bfR_\sR\tr\,\bfR_\sS)                         
\end{align}

\bigskip
In both cases, notice that even though the vector difference $\bftheta$ is typically supposed to be small, the definitions above hold for any value of $\bftheta$ (up to the first coverage of the $SO(3)$ manifold, that is, for angles $\theta<\pi$).

\subsection{The four possible derivative definitions}



\subsubsection{Functions from vector space to vector space}

The scalar and vector cases follow the classical definition of the derivative: given a function $f:\bbR^m\to\bbR^n$, we use $\{+,-\}$ to define the derivative as
%
\begin{align}
\dpar{f(\bfx)}{\bfx} &\te \lim_{\delta\bfx\to0}\frac{f(\bfx+\delta\bfx)-f(\bfx)}{\delta\bfx} &&\in \bbR^{n\times m} \label{equ:derivative_vector}
\end{align}
%
Euler integration produces linear expressions of the form
%
\begin{align*}
f(\bfx+\Delta\bfx) &\approx f(\bfx) + \dpar{f(\bfx)}{\bfx}\Delta\bfx
& \in \bbR^n
\end{align*}

\subsubsection{Functions from $SO(3)$ to $SO(3)$}

Given a function $f:SO(3) \to SO(3)$ with $\sR\in SO(3)$ and a local, small angular variation $\bth\in\bbR^3$, we use $\{\oplus,\ominus\}$ to define the derivative as
%
\begin{align}
\dpar{f(\sR)}{\bth} 
&\te \lim_{\delta\bth\to0}\frac{f(\sR\oplus\delta\bth)\ominus f(\sR)}{\delta\bth}  && \in \bbR^{3\times 3}\\
&= \lim_{\delta\bth\to0}\frac{\Log\big(f\inv(\sR)\,f(\sR\Exp(\delta\bth))\big)}{\delta\bth} \label{equ:derivative_SO3}
\end{align}
%
Euler integration produces expressions of the form,
%
\begin{align*}
f(\sR\oplus\Delta\bth) &\approx f(\sR)\,\oplus\,\dpar{f(\sR)}{\bth}\,\Delta\bth
 \te f(\sR)\Exp\left(\dpar{f(\sR)}{\bth}\Delta\bth\right)
 & \in SO(3)
\end{align*}




\subsubsection{Functions from vector space to $SO(3)$}

For the case of a function $f:\bbR^m\to SO(3)$, we use `+' for the vector perturbations, and `$\ominus$' for the $SO(3)$ difference,
%
\begin{align}
\dpar{f(\bfx)}{\bfx} &\te \lim_{\delta\bfx\to0} \frac{ f(\bfx+\delta\bfx)\ominus f(\bfx)}{\delta\bfx} && \in \bbR^{3\times m} \label{equ:dif_RtoSO3}\\
&= \lim_{\delta\bfx\to0} \frac{\Log(f\inv(\bfx) f(\bfx+\delta\bfx))}{\delta\bfx}
\end{align}
%
Euler integration produces expressions of the form,
%
\begin{align*}
f(\bfx+\Delta\bfx) &\approx f(\bfx)\,\oplus\,\dpar{f(\bfx)}{\bfx}\,\Delta\bfx
 \te f(\bfx)\,\Exp\left(\dpar{f(\bfx)}{\bfx}\Delta\bfx\right)
 & \in SO(3)
\end{align*}

\subsubsection{Functions from $SO(3)$ to vector space}

For the case of a function $f: SO(3)\to\bbR^n$, we use `$\oplus$' for the $SO(3)$ perturbations, and `$-$' for the vector difference,
%
\begin{align}
\dpar{f(\sR)}{\bth} &\te \lim_{\delta\bth\to0} \frac{f(\sR\oplus\delta\bth) - f(\sR)}{\delta\bth} && \in \bbR^{n\times 3} \label{equ:jacobian_SO3_Rn}\\
&= \lim_{\delta\bth\to0} \frac{f(\sR\Exp(\delta\bth)) - f(\sR)}{\delta\bth}
\end{align}
%
Euler integration produces expressions of the form,
%
\begin{align*}
f(\sR\oplus\delta\bth) &\approx f(\sR)+\dpar{f(\sR)}{\bth}\,\Delta\bth
 \te f(\sR)+\Exp\left(\dpar{f(\sR)}{\bth}\Delta\bth\right)
 & \in SO(3)
\end{align*}


\subsection{Right Jacobian of $SO(3)$ }

We define the right Jacobian of $SO(3)$ as, 
%
\begin{align}
\bfJ_r(\bth) &\te \dpar{\Exp(\bth)}{\bth} 
\end{align}
%
Since the exponential $\Exp()$ is an application $\bbR^3\to SO(3)$,
we implement this derivative using \eqRef{equ:dif_RtoSO3},
%
\begin{align}
\bfJ_r(\bth) &= \lim_{\dth\to0}\frac{\Exp(\bth+\dth)\ominus\Exp(\bth)}{\dth} \\
 &= \lim_{\dth\to0}\frac{\Log(\Exp(\bth)\tr\Exp(\bth+\dth))}{\dth} && \textrm{if using $\bfR$} \\
 &= \lim_{\dth\to0}\frac{\Log(\Exp(\bth)^*\ot\Exp(\bth+\dth))}{\dth} && \textrm{if using $\bfq$} 
 ~.
\end{align}
%
It has the properties, for any $\bth$ and small $\dth$,
%
\begin{align}
\Exp(\bth+\dth) &\approx \Exp(\bth)\Exp(\bfJ_r(\bth)\dth) \\
\Exp(\bth)\Exp(\dth) &\approx \Exp(\bth+\bfJ_r\inv(\bth)\,\dth) \\
\Log(\Exp(\bth)\Exp(\dth)) &\approx \bth+\bfJ_r\inv(\bth)\,\dth 
\end{align}

The right Jacobian and its inverse can be computed in closed form with
%
\begin{align}
\bfJ_r(\bth) &= \bfI - \frac{1-\cos\nth}{\nth^2}\hatx{\bth} + \frac{\nth-\sin\nth}{\nth^3}\hatx{\bth}^2 \\
\bfJ_r\inv(\bth) &= \bfI + \frac12\hatx{\bth} + \left(\frac1{\nth^2} - \frac{1+\cos\nth}{2\nth\sin\nth}\right)\hatx{\bth}^2
\end{align}






%%%%%%%%%%%%%%%%%%%%%%%%%%%%%%%%%%%%%%%%%%%%%%%%%%%%%%%%%%%%%%%%%%%%%%%%%%%
\section{Rules, do's and don'ts for Jacobians}
\label{sec:DosDonts}

We provide a collection of rules which come very handy to develop Jacobians. They come organized under helper \com{keys}\!\!\!\!, which we use to refer to each of these properties in our developments.

\subsection{Useful properties: Do's}

\paragraph{\cchain : Chain rule}

\begin{align}
\dpar{\bfz}{\bfx} = \dpar{\bfz}{\bfy}\cdot\dpar{\bfy}{\bfx}
\end{align}

\paragraph{\ccross : Cross product and skew-symmetric matrix}

\begin{align}
\hatx{\bfa}\bfb &= \bfa\times\bfb \\
\hatx{\bfa}\bfb &= -\hatx{\bfb}\bfa \\
\bfR\tr\hatx{\bfR\bfa}\bfR &= \hatx{\bfa} \\
\hatx{\bfR\bfa} &= \bfR\hatx{\bfa}\bfR\tr 
\end{align}

\paragraph{\cJr : Right Jacobian of $SO(3)$ }

It has the properties, for any $\bth$ and small $\dth$,
%
\begin{align}
\Exp(\bth+\dth) &\approx \Exp(\bth)\Exp(\bfJ_r(\bth)\dth) \\
\Exp(\bth)\Exp(\dth) &\approx \Exp(\bth+\bfJ_r\inv(\bth)\,\dth) \\
\Log(\Exp(\bth)\Exp(\dth)) &\approx \bth+\bfJ_r\inv(\bth)\dth %\\
\end{align}
%






\paragraph{\csmall : Small angle approximations}

Let $\dth$ be a small angle vector. Then,
%
\begin{align}
\Exp(\dth) &\approx \bfI + \hatx{\dth} \\
\Exp(\dth)\tr &\approx \bfI - \hatx{\dth} \\
\Exp(\dth_1)\Exp(\dth_2) &\approx \Exp(\dth_1+\dth_2) \\
\textstyle\prod_i \Exp(\dth_i) &\approx \Exp\!\big(\textstyle\sum_i\dth_i\big) \\
\bfJ_r(\dth) &\approx \bfI - \frac12\hatx{\dth} \\
\bfJ_r\inv(\dth) &\approx \bfI + \frac12\hatx{\dth} 
\end{align}
%
Example: we often use $\Exp(\bfJ_r(\bth)\dth)\approx \bfI + \hatx{\bfJ_r(\bth)\dth}$.

\paragraph{\cswap : Reversing product order}

Since $$\bfR\Exp(\bth)\bfR\tr=\bfR\exp(\hatx{\bth})\bfR\tr=\exp(\bfR\hatx{\bth}\bfR\tr)=\exp(\hatx{\bfR\bth})=\Exp(\bfR\bth),$$ then
%
\begin{align}
\Exp(\bth)\bfR &= \bfR\Exp(\bfR\tr\bth) \\
\bfR\tr\Exp(\bth)\bfR &= \Exp(\bfR\tr\bth) \\
\Exp(\bth)\Exp(\bphi) &= \Exp(\bphi)\Exp(\Exp(\bphi)\tr\bth) 
\end{align}


\paragraph{\cexpand, \csubst, \ccancel : Expand, substitute, cancel :} This happens when we expand or substitute a previously defined term, or when we cancel terms.

\paragraph{\tcom{$\oplus$}, \tcom{$\ominus$}, \tcom{(1)} : Apply definition :} This happens when we apply a particular definition or equation number.

\subsection{Common mistakes: Don'ts}

\begin{align}
\Exp(\bth_1+\bth_2) &\ne \Exp(\bth_1)\Exp(\bth_2) \\
\Log(\bfR_1\bfR_2) &\ne \Log(\bfR_1) + \Log(\bfR_2) \\
\bfJ_r(\bth) &\ne \bfI - \frac12\hatx{\bth} \\
\bfJ_r\inv(\bth) &\ne \bfI + \frac12\hatx{\bth} 
\end{align}
%
however, these hold approximately true for small angle vectors. See \csmall above.


\subsection{Examples}

\subsubsection{Example 1: $\bbR^3\times\bbR^3\to\bbR^3$} 

The rotation $f(\bth,\bfv) = \bfR(\bth)\,\bfv = \Exp(\bth)\,\bfv \in \bbR^3$ produces vectors of $\bbR^3$ from vectors $\bth,\bfv$ in $\bbR^3$. Its Jacobians are defined by \eqRef{equ:derivative_vector} and developed as
%
\begin{align*}
\dpar{\bfR(\bth)\bfv}{\bth} 
&= \lim_{\delta\bth\to0}\frac{\Exp(\bth+\delta\bth)\bfv-\Exp(\bth)\bfv}{\delta\bth} \\
\cJr
&= \lim_{\delta\bth\to0}\frac{\Exp(\bth)\Exp(\bfJ_r(\bth)\delta\bth)\bfv-\Exp(\bth)\bfv}{\delta\bth} \\
\csmall
&= \lim_{\delta\bth\to0}\frac{\Exp(\bth)(\bfI+\hatx{\bfJ_r(\bth)\delta\bth})\bfv-\Exp(\bth)\bfv}{\delta\bth} \\
\ccancel
&= \lim_{\delta\bth\to0}\frac{\Exp(\bth)\hatx{\bfJ_r(\bth)\delta\bth}\bfv}{\delta\bth} \\
\ccross
&= \lim_{\delta\bth\to0}\frac{-\Exp(\bth)\hatx{\bfv}\bfJ_r(\bth)\delta\bth}{\delta\bth} \\
&= -\bfR(\bth)\hatx{\bfv}\bfJ_r(\bth) 
\end{align*}
%
and
%
\begin{align*}
\dpar{\bfR(\bth)\bfv}{\bfv} 
&= \lim_{\partial\bfv\to0}\frac{\bfR(\bth)(\bfv+\partial\bfv)-\bfR(\bth)\bfv}{\partial\bfv} \\
\com{cancel} 
&= \bfR(\bth)
\end{align*}

\subsubsection{Example 2: $SO(3)\times\bbR^3\to\bbR^3$} 

The rotation $f(\bfR,\bfv) = \bfR\,\bfv \in \bbR^3$ produces vectors of $\bbR^3$ from elements $\bfR\in SO(3)$ and vectors in $\bbR^3$. The first Jacobian is defined by \eqRef{equ:jacobian_SO3_Rn} and developed as
%
\begin{align*}
\dpar{\bfR\bfv}{\bth} 
&\te \lim_{\delta\bth\to0}\frac{(\bfR\oplus\delta\bth)\bfv-\bfR\bfv}{\delta\bth} \\
\com{$\oplus$}
&= \lim_{\delta\bth\to0}\frac{\bfR\Exp(\delta\bth)\bfv-\bfR\bfv}{\delta\bth} \\
\csmall
&= \lim_{\delta\bth\to0}\frac{\bfR\tdot(\bfI+\hatx{\delta\bth})\bfv-\bfR\bfv}{\delta\bth} \\
\ccancel
&= \lim_{\delta\bth\to0}\frac{\bfR\hatx{\delta\bth}\bfv}{\delta\bth} \\
\ccross
&= \lim_{\delta\bth\to0}\frac{-\bfR\hatx{\bfv}\delta\bth}{\delta\bth} \\
&= -\bfR\hatx{\bfv} 
\end{align*}
%
The second Jacobian is defined by \eqRef{equ:derivative_vector} and trivially develops as,
%
\begin{align*}
\dpar{\bfR\bfv}{\bfv} 
&\te \lim_{\partial\bfv\to0}\frac{\bfR\tdot(\bfv+\partial\bfv)-\bfR\bfv}{\partial\bfv} \\
&= \bfR
\end{align*}

\subsubsection{Example 3: $SO(3)\times\bbR^3\to SO(3)$} 

The function $f(\bfR,\bw) = \bfR\Exp(\bw\dt)\in SO(3)$ produces elements of $SO(3)$ from elements in $SO(3)$ and vectors $\bw$ in $\bbR^3$. Its Jacobians are 
%
\begin{align*}
\dpar{\bfR\Exp(\bw\dt)}{\bth} 
&= \lim_{\delta\bth\to0}\frac{(\bfR\oplus\delta\bth)\Exp(\bw\dt)\ominus(\bfR\Exp(\bw\dt))}{\delta\bth} \\
\com{$\oplus,\ominus$}
&= \lim_{\delta\bth\to0}\frac{\Log\big((\bfR\Exp(\bw\dt))\inv \bfR\Exp(\delta\bth)\Exp(\bw\dt)\big)}{\delta\bth} \\
&= \lim_{\delta\bth\to0}\frac{\Log\big(\Exp(\bw\dt)\tr\bfR\tr \bfR\Exp(\delta\bth)\Exp(\bw\dt)\big)}{\delta\bth} \\
\ccancel
&= \lim_{\delta\bth\to0}\frac{\Log\big(\Exp(\bw\dt)\tr\Exp(\delta\bth)\Exp(\bw\dt)\big)}{\delta\bth} \\
\cswap
&= \lim_{\delta\bth\to0}\frac{\Log\big(\Exp(\Exp(\bw\dt)\tr\delta\bth)\big)}{\delta\bth} \\
\ccancel
&= \lim_{\delta\bth\to0}\frac{\Exp(\bw\dt)\tr\delta\bth}{\delta\bth} \\
&= \Exp(-\bw\dt) 
\end{align*}
%
and
%
\begin{align*}
\dpar{\bfR\Exp(\bw\dt)}{\bw} 
&= \lim_{\delta\bw\to0}\frac{\bfR\Exp((\bw+\delta\bw)\dt) \ominus (\bfR\Exp(\bw\dt)) }{\delta\bw} \\
\com{$\ominus$}
&= \lim_{\delta\bw\to0}\frac{\Log\big((\bfR\Exp(\bw\dt))\inv \, \bfR\Exp(\bw\dt+\delta\bw\dt)\big)}{\delta\bw} \\
\cJr
&= \lim_{\delta\bw\to0}\frac{\Log\big((\bfR\Exp(\bw\dt))\inv \bfR\Exp(\bw\dt)\Exp(\bfJ_r(\bw\dt)\delta\bw\dt)\big)}{\delta\bw} \\
\ccancel
&= \lim_{\delta\bw\to0}\frac{\Log\big(\Exp(\bfJ_r(\bw\dt)\delta\bw\dt)\big)}{\delta\bw} \\
\com{cancel}
&= \bfJ_r(\bw\dt)\dt
\end{align*}
%
%This can be seen in a easier way using the chain rule on $\bw\dt$,
%%
%\begin{align*}
%\dpar{\bfR\Exp(\bw\dt)}{\bw} 
%&= \dpar{\bfR\Exp(\bw\dt)}{\bw\dt}\dpar{\bf\dt}{\bw} \\
%&= \dpar{\bfR\Exp(\bw\dt)}{\bw\dt}\dt \\
%\cJr
%&= \bfJ_r(\bw\dt)\dt
%\end{align*}
%
Refer to the text for other developments, \eg~\secRef{sec:jac_first_delta}.



\subsubsection{Example 4: $SO(3)\times SO(3)\to SO(3)$}

The function $f(\bfR,\bfS) = \bfR\{\bftheta\}\,\bfS\{\bfphi\} \in SO(3)$ produces the concatenation of rotations, or rotation composition. Its Jacobians are
%
\begin{align*}
\dpar{\bfR\{\bftheta\}\,\bfS\{\bfphi\}}{\bftheta} 
&= \lim_{\delta\bftheta\to0}\frac{\Log\big((\bfR\bfS)\inv((\bfR\oplus\delta\bftheta)\bfS)\big)}{\delta\bftheta} \\
&= \lim_{\delta\bftheta\to0}\frac{\Log\big((\bfR\bfS)\tr(\bfR\Exp(\delta\bftheta)\bfS)\big)}{\delta\bftheta} \\
&= \lim_{\delta\bftheta\to0}\frac{\Log\big(\bfS\tr\bfR\tr\bfR\Exp(\delta\bftheta)\bfS\big)}{\delta\bftheta} \\
&= \lim_{\delta\bftheta\to0}\frac{\Log\big(\bfS\tr\Exp(\delta\bftheta)\bfS\big)}{\delta\bftheta} \\
\cswap
&= \lim_{\delta\bftheta\to0}\frac{\Log\big(\Exp(\bfS\tr\delta\bftheta)\big)}{\delta\bftheta} \\
&= \lim_{\delta\bftheta\to0}\frac{\bfS\tr\delta\bftheta}{\delta\bftheta} \\
&= \bfS\tr
\end{align*}
%
and
%
\begin{align*}
\dpar{\bfR\{\bftheta\}\,\bfS\{\bfphi\}}{\bfphi} 
&= \lim_{\delta\bfphi\to0}\frac{\Log\big((\bfR\bfS)\tr(\bfR(\bfS\oplus\delta\bfphi))\big)}{\delta\bfphi} \\
&= \lim_{\delta\bfphi\to0}\frac{\Log\big((\bfR\bfS)\tr(\bfR\bfS\Exp(\delta\bfphi))\big)}{\delta\bfphi} \\
&= \lim_{\delta\bfphi\to0}\frac{\Log\big(\Exp(\delta\bfphi)\big)}{\delta\bfphi} \\
&= \lim_{\delta\bfphi\to0}\frac{\delta\bfphi}{\delta\bfphi} \\
&= \bfI
\end{align*}





\section{Uncertainties in $SO(3)$}

\subsection{Description of uncertainty}
\subsubsection{Vector spaces}
\begin{align*}
\widetilde\bfx&=\bfx+\delta\bfx
\end{align*}
\subsubsection{$SO(3)$}
\begin{align*}
\widetilde\bfR&=\bfR\oplus\dth\te\bfR\Exp(\delta\bth) \\
\widetilde\bfq&=\bfq\oplus\dth\te\bfq\ot\Exp(\delta\bth)
\end{align*}

\subsection{Uncertainty propagation}

%
The uncertainty of the composition of two uncertain elements of $SO(3)$ can be derived as follows (we use the matrix form for convenience),
%
\begin{align*}
\widetilde\bfR = \bfR\Exp(\dth) 
&= \widetilde{\bfR_1\bfR_2} \\
&= \widetilde\bfR_1\widetilde\bfR_2 \\
\cexpand
&= \bfR_1\Exp(\dth_1)\bfR_2\Exp(\dth_2) \\
\cswap
&= \bfR_1\bfR_2\Exp(\bfR_2\tr\dth_1)\Exp(\dth_2) \\
\cJr
&\approx \bfR_1\bfR_2\Exp(\bfR_2\tr\dth_1+\bfJ_r\inv(\bfR_2\tr\dth_1)\dth_2) \\
&= \bfR\Exp(\bfR_2\tr\dth_1+\bfJ_r\inv(\bfR_2\tr\dth_1)\dth_2) 
\end{align*}
%
and therefore,
%
\begin{align}
\dth = \bfR_2\tr\dth_1+\bfJ_r\inv(\bfR_2\tr\dth_1)\dth_2
\end{align}
%
We can derive equivalent Jacobian matrices for the propagation. Since each partial derivative assumes null perturbations in  other variables, we have,
%
\begin{align}
\DTH_{\dth_1} \te \dpar{\bth}{\bth_1} &= \frac{\dth(\dth_2=0)}{\dth_1} = \bfR_2\tr \\
\DTH_{\dth_2} \te \dpar{\bth}{\bth_2} &= \frac{\dth(\dth_1=0)}{\dth_2} = \bfI
\end{align}
%
yielding
%
\begin{align}
\dth \approx \bfR_2\tr\dth_1+\dth_2
\end{align}
%
which constitutes a regular frame transformation operation. The covariances are propagated from $\bfR_1$ and $\bfR_2$ to $\bfR$ as 
%
\begin{align*}
\Sigma = \bfR_2\tr\,\Sigma_1\,\bfR_2 + \Sigma_2
\end{align*}


%%%%%%%%%%%%%%%%%%%%%%%%%%%%%%%%%%%%%%%%%%%%%%%%%%%%%%%%%%%%%%%%%%%%%%%%%%%%%%%%
%\section*{ACKNOWLEDGMENT}
%Work sponsored by FP7 EUROC, ANR LOCO3D and?

%%%%%%%%%%%%%%%%%%%%%%%%%%%%%%%%%%%%%%%%%%%%%%%%%%%%%%%%%%%%%%%%%%%%%%%%%%%%%%%%
\bibliographystyle{IEEEtran}
\bibliography{bib,slam}


\end{document}
