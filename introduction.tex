% !TEX root = main.tex

\section{Introduction}\label{sec:intro}


\cite{wensing2017proprioceptive}

\begin{figure}
\centering
	\caption{Cover figure}
	\label{fig:cover}
\end{figure}

Graphical methods have been extensively used for modeling estimation problems consisting of sparse networks of constraints. 
In robotics, the problems of visual odometry, and simultaneous localization and mapping, have reached a high degree of maturity, in great part thanks of the graphical representation. 
This is so, among other aspects, because of the power of the graphical representation to accurately model complex estimation problems. 
These often involve dynamics, proprioceptive measures, exteroceptive measures, and self-calibration. 
The graphical representation also allows for the design of powerful nonlinear estimation solvers, which can be built taking into account the needs for accuracy, robustness and CPU-performance.
