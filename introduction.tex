% !TEX root = main.tex

\section{Introduction}\label{sec:intro}

One of the most common ways to describe the trajectory of a body in space is to use odometry that can be derived from a multitude of sensors, from encoders to GPS and cameras.
The quality of this information is also known to be sensor specifications dependent and to accumulate measurement errors. The development of SLAM (Simultaneous Localization And Mapping) is today advanced enough to resolve 
this dependency over time with loop closure strategies. However, other strategies are required to comply with the particularities of legged locomotion and physical interactions in unstructured environments. These requirements can be met in 
different levels, e.g. actuator designs or control strategies.

  \begin{itemize}
   \item Actuator design  \cite{wensing2017proprioceptive}
   \item Control strategies
    \begin{itemize}
     \item Control in position
     \item Torque Control
     \item other ?
    \end{itemize}
  \end{itemize}

However, a good pose estimation of the robot in its environment is still needed for effective interactions with the environment. This estimation can then be used to compute the commands that will be sent to actuators.
Fusion strategies with information coming from different sensors can be used to achieve this goal. Considering the application in a real environment, we can fuse the odometer with a high frequency proprioceptive sensor 
to get better predictions of robot poses by improving integration points.

Using an IMU (Inertial Measurement Unit) allows us get acceleration and angular velocity measurement that can be used to compute the pose of the robot and get information similar to odometry by integration for the fusion step.
These sensors are now used in a wide range of applications : aerospace, SLAM, human motion analysis, or robotics.


\begin{description}
 \item [Paper subject] \hfill \\ \textit{Use least square optimization to overcome the estimation of the trajectory using an IMU and odometry information. Need of a low frequency odometer to re-estimate biases during motions. Feet trajectory estimation.}
\end{description}


\begin{figure}
\centering
	\caption{Cover figure}
	\label{fig:cover}
\end{figure}

Graphical methods have been extensively used for modeling estimation problems consisting of sparse networks of constraints. 
In robotics, the problems of visual odometry, and simultaneous localization and mapping, have reached a high degree of maturity, in great part thanks of the graphical representation. 
This is so, among other aspects, because of the power of the graphical representation to accurately model complex estimation problems. 
These often involve dynamics, proprioceptive measures, exteroceptive measures, and self-calibration. 
The graphical representation also allows for the design of powerful nonlinear estimation solvers, which can be built taking into account the needs for accuracy, robustness and CPU-performance.
