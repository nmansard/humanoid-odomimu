% !TEX root = main.tex

%%%%%%%%%%%%%%%%%%%%%%%%%%%%%%%%%%%%%%%%%%%%%%%%%%%%%%%%%%%%
\appendices

\section{Definition of the derivatives in $SO(3)$}
\label{sec:derivatives_SO3}

\subsection{The additive and subtractive operators in $SO(3)$}

%In vector spaces $\bbR^n$, the addition and subtraction operations are performed with the regular sum `$+$' and minus `$-$' operations.
%In $SO(3)$ this is not possible, but equivalent operators are needed for establishing a proper calculus corpus. 
%
%We thus define the plus and minus operators, $\oplus,\ominus$, between elements $\sR\in SO(3)$, and elements $\bth\in\bbR^3$ of the tangent space at $\sR$, as follows.

\paragraph{The plus operator.}
The `plus' operator $\oplus:SO(3)\times\bbR^3\to SO(3)$ produces an element $\sS$ of $SO(3)$ which is the result of composing a reference element $\sR$ of $SO(3)$ with a (often small) rotation specified by a vector of $\bth\in\bbR^3$ in the vector space tangent to the $SO(3)$ manifold at the reference element $\sR$,
%
\begin{align}
\sS = \sR\oplus \bth &\te \sR\circ\Exp(\bth) && \sR,\sS\in SO(3),~ \bth\in\bbR^3 
\end{align}
%
Notice that this operator may be defined for any representation of $SO(3)$. In particular, %for the quaternion and rotation matrix we have,
%
\begin{align}
\bfq\oplus\bth &= \bfq\ot\Exp(\bth) \\
\bfR\oplus \bth &= \bfR\Exp(\bth) 
\end{align}

\paragraph{The minus operator.}
The `minus' operator $\ominus:SO(3)\times SO(3)\to\bbR^3$ is the inverse of the above. It returns the vectorial angular difference $\bth\in\bbR^3$ between two elements of $SO(3)$. This difference is expressed in the  vector space tangent to the manifold at the reference element $\sR$, 
%
\begin{align}
\bth=\sS\ominus \sR
&\te \Log(\sR\inv \circ \sS)     && \sR,\sS\in SO(3),~ \bth\in\bbR^3  
\end{align}
%
and in particular,
%which for the quaternion and rotation matrix reads,
%
\begin{align}
\bth &= \bfq\ominus\bfp = \Log(\bfp^*\ot\bfq)                      \\
\bth &= \bfS\ominus\bfR = \Log(\bfR\tr\,\bfS)                         
\end{align}

%\bigskip
%In both cases, notice that even though the vector difference $\bftheta$ is typically supposed to be small, the definitions above hold for any value of $\bftheta$ (up to the first coverage of the $SO(3)$ manifold, that is, for angles $\norm{\bth}<\pi$).

\subsection{The four possible derivative definitions}



\subsubsection{Functions from vector space to vector space}

We use the standard operators $\{+,-\}$ to define the derivative as
%
\begin{align}
\dpar{f(\bfx)}{\bfx} &\te \lim_{\delta\bfx\to0}\frac{f(\bfx+\delta\bfx)-f(\bfx)}{\delta\bfx} &&\in \bbR^{n\times m} \label{equ:derivative_vector}
\end{align}

\subsubsection{Functions from $SO(3)$ to $SO(3)$}

We use $\{\oplus,\ominus\}$ to define the derivative as
%
\begin{align}
\dpar{f(\sR)}{\bth} 
&\te \lim_{\delta\bth\to0}\frac{f(\sR\oplus\delta\bth)\ominus f(\sR)}{\delta\bth}  && \in \bbR^{3\times 3}
%\\
%&= \lim_{\delta\bth\to0}\frac{\Log\big(f\inv(\sR)\,f(\sR\Exp(\delta\bth))\big)}{\delta\bth} 
\label{equ:derivative_SO3}
\end{align}

\subsubsection{Functions from vector space to $SO(3)$}

We use `+' for the vector perturbations, and `$\ominus$' for the $SO(3)$ difference,
%
\begin{align}
\dpar{f(\bfx)}{\bfx} &\te \lim_{\delta\bfx\to0} \frac{ f(\bfx+\delta\bfx)\ominus f(\bfx)}{\delta\bfx} && \in \bbR^{3\times m} \label{equ:dif_RtoSO3}
%\\
%&= \lim_{\delta\bfx\to0} \frac{\Log(f\inv(\bfx) f(\bfx+\delta\bfx))}{\delta\bfx}
\end{align}
%
%Euler integration produces expressions of the form,
%%
%\begin{align*}
%f(\bfx+\Delta\bfx) &\approx f(\bfx)\,\oplus\,\dpar{f(\bfx)}{\bfx}\,\Delta\bfx
% \te f(\bfx)\,\Exp\left(\dpar{f(\bfx)}{\bfx}\Delta\bfx\right)
% & \in SO(3)
%\end{align*}

\subsubsection{Functions from $SO(3)$ to vector space}

%For the case of a function $f: SO(3)\to\bbR^n$, we 
We use `$\oplus$' for the $SO(3)$ perturbations, and `$-$' for the vector difference,
%
\begin{align}
\dpar{f(\sR)}{\bth} &\te \lim_{\delta\bth\to0} \frac{f(\sR\oplus\delta\bth) - f(\sR)}{\delta\bth} && \in \bbR^{n\times 3} \label{equ:jacobian_SO3_Rn}
%\\
%&= \lim_{\delta\bth\to0} \frac{f(\sR\Exp(\delta\bth)) - f(\sR)}{\delta\bth}
\end{align}
%
%Euler integration produces expressions of the form,
%%
%\begin{align*}
%f(\sR\oplus\delta\bth) &\approx f(\sR)+\dpar{f(\sR)}{\bth}\,\Delta\bth
% \te f(\sR)+\Exp\left(\dpar{f(\sR)}{\bth}\Delta\bth\right)
% & \in SO(3)
%\end{align*}


\subsection{Right Jacobian of $SO(3)$ }

We define the right Jacobian of $SO(3)$ as, 
%
\begin{align}
\bfJ_r(\bth) &\te \dpar{\Exp(\bth)}{\bth} 
\end{align}
%
Since the exponential $\Exp()$ is an application $\bbR^3\to SO(3)$,
we implement this derivative using \eqRef{equ:dif_RtoSO3}.
%
%\begin{align}
%\bfJ_r(\bth) &= \lim_{\dth\to0}\frac{\Exp(\bth+\dth)\ominus\Exp(\bth)}{\dth} \\
% &= \lim_{\dth\to0}\frac{\Log(\Exp(\bth)\tr\Exp(\bth+\dth))}{\dth} && \textrm{if using $\bfR$} \\
% &= \lim_{\dth\to0}\frac{\Log(\Exp(\bth)^*\ot\Exp(\bth+\dth))}{\dth} && \textrm{if using $\bfq$} 
% ~.
%\end{align}
%
The right Jacobian and its inverse can be computed in closed form with
%
\begin{align}
\bfJ_r(\bth) &= \bfI - \frac{1-\cos\nth}{\nth^2}\hatx{\bth} + \frac{\nth-\sin\nth}{\nth^3}\hatx{\bth}^2 \\
\bfJ_r\inv(\bth) &= \bfI + \frac12\hatx{\bth} + \left(\frac1{\nth^2} - \frac{1+\cos\nth}{2\nth\sin\nth}\right)\hatx{\bth}^2
\end{align}






%%%%%%%%%%%%%%%%%%%%%%%%%%%%%%%%%%%%%%%%%%%%%%%%%%%%%%%%%%%%%%%%%%%%%%%%%%%
\subsection{Useful properties for Jacobian development}
\label{sec:DosDonts}

We provide a collection of rules which come very handy to develop Jacobians. They come organized under helper \com{keys}\!\!\!\!, which we use to refer to each of these properties in our developments.

%\subsection{Useful properties: Do's}

%\paragraph{\cchain : Chain rule}
%
%\begin{align}
%\dpar{\bfz}{\bfx} = \dpar{\bfz}{\bfy}\cdot\dpar{\bfy}{\bfx}
%\end{align}

\paragraph{\ccross : Cross product and skew-symmetric matrix}
%
\begin{align}
\hatx{\bfa}\bfb &= -\hatx{\bfb}\bfa 
\\
\hatx{\bfR\bfa} &= \bfR\hatx{\bfa}\bfR\tr 
\end{align}

\paragraph{\cJr : Right Jacobian of $SO(3)$ }

It has the properties, for any $\bth$ and small $\dth$,
%
\begin{align}
\Exp(\bth+\dth) &\approx \Exp(\bth)\Exp(\bfJ_r(\bth)\dth) \\
\Exp(\bth)\Exp(\dth) &\approx \Exp(\bth+\bfJ_r\inv(\bth)\,\dth) 
\end{align}
%






\paragraph{\csmall : Small angle approximations}

Let $\dth$ be a small angle vector. Then,
%
\begin{align}
\Exp(\dth) &\approx \bfI + \hatx{\dth} \\
%\Exp(\dth)\tr &\approx \bfI - \hatx{\dth} \\
%\Exp(\dth_1)\Exp(\dth_2) &\approx \Exp(\dth_1+\dth_2) \\
%\textstyle\prod_i \Exp(\dth_i) &\approx \Exp\!\big(\textstyle\sum_i\dth_i\big) \\
\bfJ_r(\dth) &\approx \bfI - \frac12\hatx{\dth} 
%\\
%\bfJ_r\inv(\dth) &\approx \bfI + \frac12\hatx{\dth} 
\end{align}
%%
%Example: we often use $\Exp(\bfJ_r(\bth)\dth)\approx \bfI + \hatx{\bfJ_r(\bth)\dth}$.

%\paragraph{\cswap : Reversing product order}
%
%Since 
%$$
%\bfR\Exp(\bth)\bfR\tr 
%%= \bfR\exp(\hatx{\bth})\bfR\tr
%= \exp(\bfR\hatx{\bth}\bfR\tr)
%%= \exp(\hatx{\bfR\bth})
%= \Exp(\bfR\bth),
%$$ 
%then
%%
%\begin{align}
%\Exp(\bth)\bfR &= \bfR\Exp(\bfR\tr\bth) \\
%\bfR\tr\Exp(\bth)\bfR &= \Exp(\bfR\tr\bth) \\
%\Exp(\bth)\Exp(\bphi) &= \Exp(\bphi)\Exp(\Exp(\bphi)\tr\bth) 
%\end{align}


\paragraph{\cexpand, \csubst, \ccancel : Expand, substitute, cancel} This happens when we expand or substitute a previously defined term, or when we cancel terms.
%
%\paragraph{\tcom{$\oplus$}, \tcom{$\ominus$}, \tcom{(1)} : Apply definition :} This happens when we apply a particular definition or equation number.

%\subsection{Common mistakes: Don'ts}
%
%\begin{align}
%\Exp(\bth_1+\bth_2) &\ne \Exp(\bth_1)\Exp(\bth_2) \\
%\Log(\bfR_1\bfR_2) &\ne \Log(\bfR_1) + \Log(\bfR_2) \\
%\bfJ_r(\bth) &\ne \bfI - \frac12\hatx{\bth} \\
%\bfJ_r\inv(\bth) &\ne \bfI + \frac12\hatx{\bth} 
%\end{align}
%%
%however, these hold approximately true for small angle vectors. See \csmall above.


\subsection{Examples}



\subsubsection{Function $SO(3)\times\bbR^3\to SO(3)$} 
\label{sec:jac_R3toSO3}

The function $f(\sR,\bw) = \bfq\od\Exp(\bw\dt) = \bfR\Exp(\bw\dt)\in SO(3)$ produces elements of $SO(3)$ from elements $\sR\in SO(3)$ and vectors $\bw\in\bbR^3$. 
Its Jacobian \wrt $\bw$ is defined by \eqRef{equ:dif_RtoSO3} and develops as,
%
\begin{align*}
\dpar{\bfR\Exp(\bw\dt)}{\bw} 
&= \lim_{\delta\bw\to0}\frac{\bfR\Exp((\bw+\delta\bw)\dt) \ominus (\bfR\Exp(\bw\dt)) }{\delta\bw} \\
\com{$\ominus$}
&= \lim_{\delta\bw\to0}\frac{\Log\big((\bfR\Exp(\bw\dt))\inv \, \bfR\Exp(\bw\dt+\delta\bw\dt)\big)}{\delta\bw} \\
\cJr
&= \lim_{\delta\bw\to0}\frac{\Log\big((\bfR\Exp(\bw\dt))\inv \bfR\Exp(\bw\dt)\Exp(\bfJ_r(\bw\dt)\delta\bw\dt)\big)}{\delta\bw} \\
\ccancel
&= \lim_{\delta\bw\to0}\frac{\Log\big(\Exp(\bfJ_r(\bw\dt)\delta\bw\dt)\big)}{\delta\bw} \\
\com{cancel}
&= \bfJ_r(\bw\dt)\dt
\end{align*}
%


\subsubsection{Function $SO(3)\times\bbR^3\to\bbR^3$} 
\label{sec:jac_SO3xR3toR3}

The rotation $f(\sR,\bfv) = \bfq\od\bfv = \bfR\,\bfv \in \bbR^3$ produces vectors of $\bbR^3$ from elements $\sR\in SO(3)$ and vectors $\bfv\in\bbR^3$. The first Jacobian is defined by \eqRef{equ:jacobian_SO3_Rn} and developed as
%
\begin{align*}
\dpar{\bfq\od\bfv}{\bth} = \dpar{\bfR\bfv}{\bth} 
&\te \lim_{\delta\bth\to0}\frac{(\bfR\oplus\delta\bth)\bfv-\bfR\bfv}{\delta\bth} \\
\com{$\oplus$}
&= \lim_{\delta\bth\to0}\frac{\bfR\Exp(\delta\bth)\bfv-\bfR\bfv}{\delta\bth} \\
\csmall
&= \lim_{\delta\bth\to0}\frac{\bfR\tdot(\bfI+\hatx{\delta\bth})\bfv-\bfR\bfv}{\delta\bth} \\
\ccancel
&= \lim_{\delta\bth\to0}\frac{\bfR\hatx{\delta\bth}\bfv}{\delta\bth} \\
\ccross
&= \lim_{\delta\bth\to0}\frac{-\bfR\hatx{\bfv}\delta\bth}{\delta\bth} \\
&= -\bfR\hatx{\bfv} 
\end{align*}
%
The second  is defined by \eqRef{equ:derivative_vector} and trivially develops as,
%
\begin{align*}
\dpar{\bfq\od\bfv}{\bfv} = \dpar{\bfR\bfv}{\bfv} 
&\te \lim_{\partial\bfv\to0}\frac{\bfR\tdot(\bfv+\partial\bfv)-\bfR\bfv}{\partial\bfv} 
= \bfR
\end{align*}





