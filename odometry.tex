% !TEX root = main.tex

\section{Graph-based inertial-kinematic odometry}

We describe the inertial-kinematic odometry for legged robots, based on a graph model. 

\subsection{Graph-based estimation through non-linear least squares optimization}

%%%%%%%%%%%%%%%%%%%%%%%%%%%%%%%%%%%%%%%%%%%%%%%%%%%%%%%%

In graph-based optimization, the problem is represented as a graph, where the nodes refer to the variables and the factors (or edges) represent the geometrical constraints between variables, produced by the measurements.
%
The state $\bfx$ is modeled as a multi-variate Gaussian distribution, and includes, in our case (see \figRef{fig:factor_graph}), foot poses and velocities $(\bfp,\bfq,\bfv)$ and IMU biases $(\bfa_b,\bw_b)$ at selected keyframes (KF) along the trajectory.
%
For each factor, we can define an error or residual $\bfr$ as the discrepancy between a measurement $\bfz$ and its expectation given the involved state variables,
%
\begin{equation}
    \bfr(\bfx) = h(\bfx) + \bfv - \bfz, \qquad \bfv \sim \mathcal \cN (0, \bfOmega\inv)\label{eq:error}
\end{equation}
%
being $h(\bfx)$ the sensor measurement model and $\bfOmega$ the information matrix of the measurement Gaussian noise $\bfv$.
Importantly, the functions $h(\bfx)$ and $\bfr(\bfx)$ are very sparse, since only a small handful of blocks of $\bfx$ are involved in each factor, which results in a loosely connected graph.
In case of variables defined in manifolds, such as quaternions or rotation matrices, (\ref{eq:error}) becomes $\bfr(\bfx) = (h(\bfx) \oplus \bfv) \ominus \bfz$, with $\bfJ_k = \partial (h_k(\bfx)\ominus\bfz_k)/\partial \Delta \bfx$. 
The $\oplus$ and $\ominus$ are the addition and subtraction operators on the manifold, as required for certain state blocks intervening in the factor (see \eg~\cite{Smith_arv90}, and \eqRef{equ:imu_residual} in \secRef{sec:imu}).

The maximum a posteriori estimation is obtained by iteratively minimizing the Mahalanobis squared norm of all linearized errors
%
\begin{align}
  \Delta \bfx^* &= \argmin_{\Delta \bfx} \sum_k \norm{ \bfr_k(\breve\bfx) + \bfJ_k \Delta \bfx }_{\bfOmega_k^{-1}}^2 \label{eq:LeastSquares}
\end{align}
%
being $\breve\bfx$ the state estimate at the current iteration, and $\bfJ_k$ the Jacobian of the $k$-th residual $\bfr_k(\bfx)$.
%
Current methods use Cholesky \cite{Kummerle_icra11,ila_ijrr17} or QR  \cite{Dellaert_ijrr06,Kaess_ijrr11} matrix factorizations to solve for $\Delta\bfx^*$, which is used to update state. 
The process is iterated until convergence.
%
Incremental methods \cite{Kaess_ijrr11,ila_ijrr17}, update the problem directly on the factorized matrix, obtaining important speed-ups.


%%%%%%%%%%%%%%%%%%%%%%%%%%%%%%%%%%%%%%%%%%%%%%%%%%%%%%%%%

%Estimation is realized with a tool in ongoing development \textbf{(cite WOLF ?)} and using Google Ceres non-linear least squares optimizer. This optimization is run on data given by the front-end part of the framework,
%which aim is to formalize the problem according to the graph-based model for aforementioned reasons. This representation, illustrated in \figRef{fig:factor_graph}, 
%gives a human-readable way to easily visualize optimized states (circles) linked to constraining information during the optimization process (factors drawn as squares). Constraints are set by the front-end process using measurements from sensors.
%Thus, it is possible to constrain states with several factors provided by different sensors leading to a fusion. Factor play a key role since they are used on top of previous states to predict next ones and compute the error with the current states.
%The optimal states are computed so that the overall residual is minimized in a least squares meaning.

\subsection{The inertial-kinematic odometry}
\subsubsection{Inertial pre-integration}
\subsubsection{Graphical model}

\begin{figure}[tb]
\begin{center}
\includegraphics[scale=0.65]{figures/graph_exploded}
%\par\vspace{4mm}
%\includegraphics[scale=0.65]{figures/graph_simplified}
\par\vspace{4mm}
\includegraphics[scale=0.65]{figures/graph_essential}
\caption{
{\bf Top}: Detailed factor graph for the initial keyframe and two steps. \emph{Circles}: state blocks for position ($\bfp$), orientation quaternion ($\bfq$), velocity ($\bfv$), accelerometer bias ($\bfa_b$), gyrometer bias ($\bw_b$). \emph{Orange}: initial pose factor. \emph{Red}: leg kinematic factor. \emph{Green}: IMU's delta pre-integration factor. 
\emph{Blue}: bias drift factor. \emph{Cyan}: bias absolute factor. \emph{Purple}: zero-velocity factor. 
{\bf Bottom}: Equivalent factor graph exhibiting one aggregate state block $\bfx=(\bfp,\bfq,\bfv,\bfa_b,\bw_b)$ for each key-frame. All factors are affecting exactly the same variables as in the Top graph.
}
\label{fig:factor_graph}
\end{center}
\end{figure}

%\begin{figure}[htbp]
%\begin{center}
%\includegraphics[scale=0.65]{figures/graph_simplified}
%\caption{default}
%\label{default}
%\end{center}
%\end{figure}

